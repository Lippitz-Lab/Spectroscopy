

%\renewcommand{\lastmod}{April 29, 2020}

\chapter{Discrete dipole approximation}


The idea is to describe a larger particle as arrangement of many small particles at positions $\mathbf{r}_i$.
Each small particle is modelled as sphere  with a  polarizability $\alpha_i$ given by the material properties and the volume of the small particle. We want to calculate the absorption, scattering and extinction spectrum of this  ensemble of small particles.


\section{Radiating electric dipole}

Let us first look at a single electric dipole $\boldsymbol{\mu}$ at the position $\mathbf{r}_0$. Its field at the position $\mathbf{r}$ is given by\footcite[eq. 8.52]{Novotny-Hecht2012} 
\begin{equation}
\mathbf{E}(\mathbf{r}) = \frac{k^2}{\epsilon_0 \, \epsilon_{out}} \, \mathbf{G}(\mathbf{r}, \mathbf{r}_0) \,  \boldsymbol{\mu}
\end{equation}
with the length $k$ of the wave vector in the medium of dielectric function $\epsilon_{out}$.
The Greens function $\mathbf{G}$ is given by\sidenote{This follows \cite{Novotny-Hecht2012}  eq. 8.55 and differs by $4 \pi k^2$ from eq. 2 in \cite{Yurkin07}}
\begin{equation}
\mathbf{G}(\mathbf{r}, \mathbf{r}_0) = \frac{e^{i k R} }{4 \pi \, k^2 \, R^3 } 
\left[  
 \left( k^2 R^2 + i k R  - 1 \right) \mathbf{1}  +    
  \left( 3 - 3 i k R - k^2 R^2  \right) \frac{\mathbf{RR}}{R^2}   
  \right]
\end{equation}
with $\mathbf{R} = \mathbf{r} - \mathbf{r}_0$, $R = |\mathbf{R}|$, $\mathbf{1}$ the unity $3 \times 3$-tensor, and $\mathbf{RR}$ the outer product of $\mathbf{R}$ with itself, i.e.
\begin{equation}
\mathbf{RR} = 
\begin{pmatrix}
R_x R_x &  R_x R_y & R_x R_z \\
R_y R_x &  R_y R_y & R_y R_z \\
R_z R_x &  R_z R_y & R_z R_z \\
\end{pmatrix} \quad .
\end{equation}


\section{Scattering sphere}

Now we replace the radiating dipole by a scattering sphere. The incident electric field $\mathbf{E}_{inc}$ induces a polarisation $\mathbf{p}$ 
\begin{equation}
\mathbf{p} = \epsilon_0 \, \epsilon_{out} \, \alpha \, \mathbf{E}_{inc}
\end{equation}
with the dielectric function $\epsilon_{out}$ of the embedding medium and the polarisability $\alpha$ of the sphere
\begin{equation}
 \alpha = 3V \, \frac{\epsilon_{in} - \epsilon_{out}}{\epsilon_{in} + 2 \epsilon_{out}}
\end{equation}
where $V$ is the volume of the sphere and  $\epsilon_{in}$ the dielectric function of it. The sphere  radiates a scattered field $\mathbf{E}_S$
\begin{align}
\mathbf{E}_S(\mathbf{r}) & =  \frac{k^2}{\epsilon_0 \, \epsilon_{out}} \, \mathbf{G}(\mathbf{r}, \mathbf{r}_0) \,  \mathbf{p} \\
 & =  \frac{1}{4 \pi \, \epsilon_0 \, \epsilon_{out}  }  \frac{e^{i k R} }{  R^3 } 
\left[  \dots \right] \, \mathbf{p}
\end{align}
where the contents of the square brackets is the same as above. $k$ is the length of the wave vector in a medium with dielectric function $\epsilon_{out}$.

\section{Multiple particles}

When we have more than one particle, each particle $i$ sees the incident field $\mathbf{E}_{i, inc}$ at the   position $\mathbf{r}_i$ plus the sum over all scatters fields $\mathbf{E}_{j, S}$ from all the other induced dipoles $j$
\begin{equation}
\mathbf{E}_{i, loc} = \mathbf{E}_{i, inc} + \sum_{j \neq i} \mathbf{E}_{j, S}
 = \mathbf{E}_{0} \, e^{i \mathbf{k} \cdot \mathbf{r}_i} \, +  \, 
 \sum_{j \neq i} \frac{k^2}{\epsilon_0 \, \epsilon_{out}} 
 \mathbf{G}(\mathbf{r}_i, \mathbf{r}_j) \,  \mathbf{p}_j  \label{eq:dda_elocal}
\end{equation}
with the dipole moment $ \mathbf{p}_j$ of the particle at position $\mathbf{r}_j$. The position of the 'receiving' particle $\mathbf{r}_i$ takes the role of $\mathbf{r}$ in the Greens function; the position of the scattering particle $\mathbf{r}_j$ takes the role of the dipole at position $\mathbf{r}_0$ above.

The local field $\mathbf{E}_{i, loc}$ then induces a dipole moment again 
\begin{equation}
\mathbf{p}_i = \epsilon_0 \, \epsilon_{out} \, \alpha_i \, \mathbf{E}_{i,loc}
\end{equation}
Both equations together form a coupled equation system for the 
 dipole moments $ \mathbf{p}_i$
\begin{equation}
\mathbf{E}_{0} \, e^{i \mathbf{k} \cdot \mathbf{r}_i} =
\frac{1}{\epsilon_0 \epsilon_{out} \alpha_i} \, \mathbf{p}_i 
 \,  - \, 
 \sum_{j \neq i} \frac{k^2}{\epsilon_0 \, \epsilon_{out}} 
 \mathbf{G}(\mathbf{r}_i, \mathbf{r}_j)  \, \mathbf{p}_j 
\end{equation} 
 which can be written as
 \begin{equation}
 \mathbf{E}_{inc} = \mathbf{A} \, \mathbf{p}
 \end{equation}
 where $\mathbf{p}$ and $ \mathbf{E}_{inc} $ are column vectors containing the induced dipole moment and the incident field of all dipoles and $\mathbf{A} $ is an interaction matrix. Its elements are $3 \times 3$-sub-matrices given by\sidenote{This assume an isotropic polarisabilty. Otherwise the diagonal elements should be $1/\alpha_{x,y,z}$ instead of $\mathbf{1}  / \alpha$.}
 \begin{align}
 \mathbf{A}_{ii} = &\frac{1}{\epsilon_0 \epsilon_{out} \alpha_i} \, \mathbf{1} \\
 \mathbf{A}_{ij} = & - \, 
 \frac{k^2}{\epsilon_0 \, \epsilon_{out}} 
 \mathbf{G}(\mathbf{r}_i, \mathbf{r}_j) 
 \end{align}
Some publications put the minus sign of the last equation into the Greens function. 

\section{Absorption, scattering and extinction} 
 
The extinction cross section can be calculated by the optical theorem from the interference of the forward-scattered wave with the incident wave. We get\footcite{Draine88,Yurkin07}
\begin{equation}
\sigma_{ext} = \frac{k}{\epsilon_0 \epsilon_{out}  \, |\mathbf{E}_{inc}|^2} \, \sum_i \, \Im ( \mathbf{p}_i \cdot \mathbf{E}_{i, inc}^\star )
\end{equation}
 The absorption can be calculated from field acting on each dipole\footcite{Yurkin07}, i.e., replacing $\mathbf{E}_{i, inc} $ by $\mathbf{E}_{i, loc} $ 
 \begin{equation}
\sigma_{abs} = \frac{k}{\epsilon_0 \epsilon_{out}  \, |\mathbf{E}_{0}|^2} \, \sum_i \, \Im ( \mathbf{p}_i \cdot \mathbf{E}_{i, loc}^\star )
\end{equation}
The scattering cross section is the difference of both
\begin{equation}
\sigma_{scat} = \sigma_{ext}  - \sigma_{abs}
\end{equation}
 
 
\section{Improvements: Radiation reaction} 

When we would have only a single dipole, then $\sigma_{scat}$ as defined above would vanish. This problem finds its roots in our definition of the polarisability $\alpha$. We need to take the radiation reaction into account\sidenote{see chapter 8.4.2. in \cite{Novotny-Hecht2012}}. When we call the 'old' definition $\alpha^{CM}$, as Clausius-Mossotti, then we define\sidenote{here in SI, in contrast to \cite{Yurkin07}} 
\begin{equation}
\alpha^{RR} = \frac{\alpha^{CM}}{1 - \frac{i k^3}{6 \pi} \alpha^{CM}}
\approx
\alpha^{CM} -  \frac{i k^3}{6 \pi} \left(\alpha^{CM}\right)^2
\end{equation}

Along the same lines, an improved relation for the absorption cross-section should be used\sidenote{see \cite{Draine88}  for a derivation, and \cite{Yurkin07} for a discussion}
 \begin{equation}
\sigma_{abs} = \frac{k}{\epsilon_0 \epsilon_{out}  \, |\mathbf{E}_{0}|^2} \, \sum_i \, \Im \left( \mathbf{p}_i \cdot \left( \frac{\mathbf{p}_i}{\epsilon_0 \epsilon_{out}  \, \alpha_{RR}} \right) ^\star \right) 
- \frac{2}{3} \, \frac{1}{4 \pi \epsilon_0 \epsilon_{out}} \,  k^3 \, |  \mathbf{p}_i |^2
\end{equation}


\section{Polarisability of  a Lorentz oscillator}

We started above with the polarisability $\alpha$ of a sphere
\begin{equation}
 \alpha = 3V \, \frac{\epsilon_{in} - \epsilon_{out}}{\epsilon_{in} + 2 \epsilon_{out}}
\end{equation}
We could also assume a Lorentz oscillator, taking into account our definition 
 \begin{equation}
\mathbf{p} = \epsilon_0 \, \epsilon_{out} \, \alpha_L \, \mathbf{E}_{inc} = e \, \mathbf{x}
\end{equation}
With this we find 
 \begin{equation}
 \alpha_L = \frac{e^2 }{ \epsilon_0 \, \epsilon_{out}  \, m} \,  \frac{1}{\omega_0^2 - \omega^2 +2  i \gamma \omega } 
\end{equation}

\section{Lattice sum}

Things become easier when we are interested in infinite lattices of identical scatterers. As we are on a lattice, all lattice points are equal, especially in the amplitude and vectorial direction $\mathbf{\hat{n}}$ of the local field. It is then 
 convenient to re-arrange eq. \ref{eq:dda_elocal}
\begin{equation}
\mathbf{E}_{i, loc} =\mathbf{E}_{0} \, e^{i \mathbf{k} \cdot \mathbf{r}_i} \, +  \, 
 \sum_{j \neq i} k^2 \, 
\mathbf{G}(\mathbf{r}_i, \mathbf{r}_j) \,    \alpha \, \mathbf{E}_{j,loc}
\end{equation}
or
\begin{equation}
E_{i, loc}  \, e^{-i \mathbf{k} \cdot \mathbf{r}_i} =\mathbf{\hat{n}} \cdot \mathbf{E}_{0}  +  \, 
 \sum_{j \neq i} k^2 \, 
 \mathbf{\hat{n}} \mathbf{G}(\mathbf{r}_i, \mathbf{r}_j)  \mathbf{\hat{n}}\,    \alpha \, E_{j,loc} \, e^{-i \mathbf{k} \cdot \mathbf{r}_i} \,
\end{equation}
so that we get
\begin{equation}
\mathbf{\hat{n}} \cdot \mathbf{E}_{0} = 
E_{loc} \left( 1 -     \alpha  \,
 \sum_{j \neq i} k^2 \, 
 \mathbf{\hat{n}} \mathbf{G}(\mathbf{r}_i, \mathbf{r}_j)  \mathbf{\hat{n}}\,     \, e^{i \mathbf{k} \cdot ( \mathbf{r}_i - \mathbf{r}_j  ) } \right)
 = 
 {E}_{loc} \left( 1 -     \alpha  \, S \right)
\end{equation}
with the retared lattice sum $S$. The induced
dipole moment becomes 
\begin{equation}
\mathbf{p} = \epsilon_0 \, \epsilon_{out} \, \alpha \, \mathbf{E}_{loc} =  \epsilon_0 \, \epsilon_{out} \, \frac{\alpha}{ 1 -     \alpha  \, S } \,
  \mathbf{E}_{0} 
\end{equation} 
or  we define an effective (lattice) polarisabilty
\begin{equation}
\alpha_\text{lattice} = \frac{\alpha}{ 1 -     \alpha  \, S } 
\end{equation} 
The extinction cross section become then\sidenote{somehow a $4 \pi$ is missing here....}
\begin{equation}
\sigma_{ext} = k \, \Im(\alpha_\text{lattice}) 
\end{equation}

%-------------------


\printbibliography[segment=\therefsegment,heading=subbibliography]

%\printbibliography



