%\documentclass[a4paper,10pt]{scrartcl}          %twoside
\documentclass[a4paper,10pt]{article}
\usepackage[utf8x]{inputenc}
\usepackage[left=3cm,right=2cm,top=2.5cm,bottom=2.5cm,a4paper,includehead]{
geometry}
%opening

\title{Integral absorption of an emitter}
%\date{}
\author{}
\pagestyle{empty}


\begin{document}


\maketitle

\pagestyle{empty}
\thispagestyle{empty}

This follows Karrai / Warburton (2003). The spectral dependence of the
absorption cross section $\sigma(\omega)$ of a single emitter (atom etc.) can be
written as
\[
 \sigma(\omega) = \sigma_0 \; \frac{\Gamma^2 / 4}{\delta^2 + \Gamma^2 / 4}
\]
where $\delta = \omega - \omega_0$ is the detuning and $\Gamma$ the FWHM of the
Lorentzian.

The integral absorption is 
\[
 \sigma_i = \int \sigma(\omega) d\omega = \sigma_0 \; \pi \; \Gamma / 2
\]




The peak absorption cross section $\sigma_0$ can be written as 
\[
 \sigma_0 = \frac{e^2 \; f}{\epsilon_0 \; c \; m_0 \; n \Gamma}
\]
where $f$ is the oscillator strength, $n$ the index of refraction of the medium,
and $\Gamma$ the dephasing rate and the same as above.


If there is only spontaneous emission, then the dephasing rate $\Gamma$ is
\[
 \Gamma_{sp} = n \frac{2 \pi}{3 \lambda^2} \; \frac{e^2 \; f}{\epsilon_0 \; c\;
m_0} \quad .
\]
This results in a peak absorption cross section of
\[
 \sigma_{0,sp} =  \frac{3}{2 \pi} \;  (\lambda / n)^2 
\]
which is independent of the oscillator strength $f$.

If the dephasing rate $\Gamma$ is larger by a factor $1/\eta$ ($\eta < 1$) than
the spontaneous rate $\Gamma_{sp}$, i.e., $\Gamma = \Gamma_{sp} / \eta$ then the
peak absorption cross section is reduced:
\[
 \sigma_0 = \frac{3}{2 \pi} \;  (\lambda / n)^2 \; \eta
\]

Everything together gives
\[
 \sigma_i = \frac{3}{2 \pi} \;  (\lambda / n)^2 \; \eta \; \Gamma \;
\frac{\pi}{2}
\]
which is equal to 
\[
 \sigma_i = \frac{3}{4} \;  (\lambda / n)^2 \;  \Gamma_{sp} 
\]


Assuming a transition at 600~nm with a radiative lifetime of 10~ns, in vacuum,
one gets $\sigma_i =$ 17~nm$^2$~meV. (NB: $\Gamma$ is an angular frequency, but
$\tau = 1 /\Gamma$)

The dataset \verb|fieldonqwires11_wire1_longtime| has a peak transient
transmission value of $\Delta T /T = 3 \cdot 10^{-6}$. The probe spot has a FWHM
of 400 nm, resulting in a probe area $A = 1.13 \cdot 400^2 = 1.8 \cdot
10^5$~nm$^2$. This gives a peak $\sigma_0 = 0.55 $~nm$^2$. The integrated
spectrum gives $\sigma_i = 108$ nm$^2$ meV, which corresponds to about 5--6 of
above single emitters, when aligned with the probe beam, or 15--18 emitters,
when oriented randomly. The number gets larger if the assumed radiative lifetime
is increased.


\end{document}
