\documentclass[a4paper,10pt]{article}
\usepackage[utf8x]{inputenc}

\usepackage{amsmath}

%opening
\title{Oscillator strength, dipole moment and radiative rate}
\author{}

\begin{document}

\maketitle

All this started when looking at the oscillator strength of colloidal CdSe nanocrystals. Leistikow et al. found
a value of $f = 0.7$ by measuring radiative and nonradiative rates in a Drexhage type of experiment. The radiative
rate corresponds to about 15 ns. However, 'everyone' knows that molecules  also have an oscillator strength
just below 1, but a lifetime of about 3 ns, at a similar wavelength. That does not fit!

Following Hilborn, the oscillator strength $f$ is given by the integrated
absorption cross section $\sigma_0$ 
\begin{eqnarray}
 f_{\text{atom}} &= &  \frac{2 \epsilon_0 m c}{\pi e^2} \; \sigma_0 \\
   & = & \frac{2 \epsilon_0 m c}{\pi e^2} \; \int \sigma(\omega) \; d\omega \\
   & = & \frac{4 \epsilon_0 m c}{ e^2} \; \int \sigma(\nu) \; d\nu \\
   & = & \frac{4 \epsilon_0 m c^2}{ e^2} \; \int \sigma(\tilde{\nu}) \;
d\tilde{\nu} 
\end{eqnarray}
where $\tilde{\nu}$ is the wavenumber. The notion of an absorption cross section
assumes (implicitly) 
that the incoming electrical field is
polarized along the transition dipole moment, or that the latter has no
preferential direction.

For molecules one finds (e.e.g, in XXX CNT paper)
\begin{eqnarray}
 f_{\text{molecule}} &= &  \frac{4 \epsilon_0 m c^2}{e^2} \; 
\frac{\ln(10)}{N_A} \; \int \epsilon(\tilde{\nu}) \; d\tilde{\nu} 
\end{eqnarray}
where $\epsilon(\tilde{\nu})$ is the molar extinction coefficient (units liter /
(mole cm) ). 

Although the equations for $f_{\text{atom}}$ and  $f_{\text{molecule}}$ look
very similar, the definition of 
the oscillator strength $f$ differs. This becomes obvious when calculating the 
spontaneous emission rate $\Gamma_{\text{rad}}$. Hilborn, Karrai  and Leistikow give
\begin{equation}
 \Gamma_{\text{rad}} = \frac{2 \pi e^2 n}{3 \lambda^2 \epsilon_0  m c} \;
f_{\text{atom}} 
\end{equation}
and XXX CNT paper
\begin{equation}
 \Gamma_{\text{rad}} = \frac{2 \pi e^2 n}{3 \lambda^2 \epsilon_0  m c} \; 3 n 
\; f_{\text{molecule}} 
\end{equation}

The factor $3n$ finds its origin in the different definition of $f$. The
absorption spectrum $\epsilon(\tilde{\nu})$
is measured in a cuvette. The orientation of the molecules in not fixed.
Averaging over all directions
reduces the absorption by a factor of $1/3$.

The influence of the environment via the  index of refraction $n$ is put into the
oscillator strength, in the
case of molecules in solution. That makes sense as molecules differ by solvents.
In the 'atom' case,
the environment is added later, and separated from the property of the dipole.

From this follows that molecular oscillator strengths are smaller by a factor of
$3n$ compared to atomic
oscillator strength. Only by this the molecular $f$ stays below 1.  As Hens XXX points out, an oscillator strength above 1 is nothing special. Bottom line is that
CdSe nanocrystals have about a factor of 4 weaker oscillator strength than good dye molecules.


\end{document}
