\documentclass[margin=0mm]{standalone}
\usepackage{tikz}
\usepackage{pgfplots}
 \pgfplotsset{compat=newest}


\usepgfplotslibrary{groupplots}

\usepackage{currfile,hyperxmp,calc}
\usetikzlibrary{calc,math}
\usetikzlibrary{calc,patterns,angles,quotes}

\begin{document}



  
  
\begin{tikzpicture}
\useasboundingbox (0,-1.0) rectangle (5.2,3.2);
%
%\draw (0,-1.0) rectangle (5.2,3.2);
%

\tikzmath{\c = 25; \d = 10;}

\coordinate (A) at (1.5,0);
\coordinate (B) at (1,2);
\coordinate (C) at (4,2);

\draw[fill=black] (B) circle (0.5mm);
\draw[fill=black] (C) circle (0.5mm);

\draw[dashed, shorten >= -\c , shorten <= -\c] (A) -- (B);
\draw[dashed, shorten >= -\c , shorten <= -\c]  (C) -- (B);
\draw[dashed, shorten >= -\c , shorten <= -\c]  (A) -- (C);

\draw[thick, ->] ($(B)!\d mm!(A)$) -- ($(B)!-\d mm!(A)$) ;
\draw[thick, ->] ($(C)!\d mm !(A)$) -- ($(C)!-\d mm!(A)$) ;

\pic [draw, <->, "$\theta$", angle eccentricity=1.5] {angle = C--A--B};

\coordinate (Bt) at  ($(B)!-\d mm!(A)$) ;
\coordinate (Ct) at  ($(C)!-\d mm!(A)$) ;
\coordinate (Ct2) at  ($(C)!-\d mm!(B)$) ;

\pic [draw, <->, "$\alpha$", angle eccentricity=1.5] {angle = C--B--Bt};
\pic [draw, <->, "$\beta$", angle eccentricity=1.5] {angle = Ct2--C--Ct};


\draw[ <->,  shorten >= 0.6mm, shorten <= 0.6mm] (B) -- node[below] {$R$} (C);

\end{tikzpicture}


\end{document}