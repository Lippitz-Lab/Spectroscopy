

%\renewcommand{\lastmod}{April 29, 2020}

\chapter{Plasmon hybridization}


This reconstructs Figure 13 in V. Myroshnychenko et al., \textit{Modelling the
optical response of gold nanoparticles}, Chem. Soc. Rev., 2008, \textbf{37},
1792. 

We start with a single sphere of radius $R$. The incomming field $E_{in}$
induces a dipole $p$ 
\[
  p =  \alpha E_{in} \quad .
\]
The polarisability $\alpha$ is given by
 \[
  \alpha = 4 \pi R^3 \;  \frac{\epsilon - \epsilon_m}{\epsilon + 2 \epsilon_m} \,
 \]
where $\epsilon(\omega)$ is the dielectric function of the sphere and $\epsilon_m$ the
dielectric constant of the medium (XXX might be assumed to be 1 somewhere,
CHECK!).


Now a second sphere is placed at a center-to-center distance of $d$. The first
dipole sees the incomming field and the field scattered by the second sphere,
and the other way round. When we assume that the incomming field is either
parallel or perpendicular the vector connecting the two spheres, then
\[
  p_1 =  \alpha_1 \left( E_{in}  + D p_2 \right)
\]
where $D$ is some coupling constant
\begin{eqnarray}
  D & = & \frac{\exp( i \; k \; d)}{4 \pi \epsilon_m d^3}  A       \qquad
\text{perpendicular} \\
  D & = & \frac{\exp( i \; k \; d)}{4 \pi \epsilon_m d^3}  (A - B)   \qquad
\text{parallel} 
\end{eqnarray}
Note that $k$ is the wavevector \textit{in the medium!}. The factors $A$ and $B$
describe the near-field of the dipole and are followsing the the paper (CHECK !)
\begin{eqnarray}
 A &= &  (k \; d)^2 + i \;  k \; d  - 1       \approx   -1 \\
  B &=&  (k \; d)^2 + 3 i \;  k \; d  - 3    \approx   -3
\end{eqnarray}
where the last approximation holds for $kd \ll 1$ which is reasonable as we
assumed $kR \ll 1$ anyway.
Using that approximation we get
\begin{eqnarray}
  D & = & \frac{-1}{4 \pi \epsilon_m d^3}       \qquad \text{perpendicular} \\
  D & = & \frac{2}{4 \pi \epsilon_m d^3}    \qquad \text{parallel} 
\end{eqnarray}



We have to solve linear equation system (Check that really $D_1 = D_2$, indep. of dipol-to-d orientation)
\begin{eqnarray}
   p_1 &=&  \alpha_1 \left( E_{in}  + D p_2 \right) \\
  p_2 &=&  \alpha_2 \left( E_{in}  + D p_1 \right) 
\end{eqnarray}
which results in 
\[
  p_1=  \frac{\alpha_1 \left( 1 + \alpha_2 D \right)}{1 - \alpha_1 \alpha_2 D^2}
\; E_{in} = \alpha_{eff} \; E_{in} \quad .
\]

While $\alpha$ has a resonance at $\epsilon(\omega) = -2$, $\alpha_{eff}$ does not
resonante there anymore, as both denominator and numerator are proportional to
$\alpha^2$. (ASSUMING equal $\alpha$ !). Howevere, we get new resonances at  
$\alpha_1 \alpha_2 D^2 = 1$, more accurately at  $\text{Re}(\alpha_1 \alpha_2
D^2) = 1$. 

Assuming $\alpha_1 = \alpha_2$ the resonance condition simplifies to 
\[
  \alpha \; D = \pm 1
\]
which can be expanded to
\[
  4 \pi R^3  \frac{\epsilon - \epsilon_m}{\epsilon + 2 \epsilon_m}  \; \frac{1}{4 \pi \epsilon_m
d^3} = \pm 1 \quad \text{or} \quad \pm \frac{1}{2}
\]
The $\pm 1$-case holds for perpendicular polarisation, the case $\pm 0.5$ for
the parallel polarisation. 
This can be simplified to 
\[
\text{Re}\left\{  \left(\frac{R}{d}\right)^3  \frac{\epsilon - \epsilon_m}{\epsilon + 2 \epsilon_m} \frac{1}{\epsilon_m} \right\} = \pm 1
\quad \text{or} \quad \pm \frac{1}{2}
\]


Now we  assume a Drude model for $\epsilon(\omega)$
\[
 \epsilon(\omega) = \epsilon_{\infty} - \frac{\omega_p^2}{\omega (\omega + i \gamma)}
\quad .
\]


%---------------


\section{Dipole approximation}

We approximate each sphere of the particle pair by a dipole $\vec{p}_i$ at
position $\vec{r}_i$. Each dipole experiences the incident field
$\vec{E}^{\text{inc}}(\vec{r}_i)$ and the field scattered from the other dipole.
The sum of these two fields multiplied by the dipoles polarizability $\alpha_i$
has to give in a self-consisten way the dipole strength
\begin{equation} \label{eq:fields}
     \vec{p}_1 = \alpha_1 \left(  \vec{E}^{\text{inc}}(\vec{r}_1) +
\vec{E}^{\text{scat}}_2(\vec{r}_1) \right)
\end{equation}
and vice versa. The scattered electrical near field $\vec{E}$ of the dipole is
\begin{equation}
  \vec{E}^{\text{scat}}_i(\vec{r}_j) = \frac{1}{4 \pi \;  \epsilon_0 \;
\epsilon_m} \; \frac{1}{\left| \Delta r \right|^3} 
          \left( \frac{3 \vec{\Delta r} \vec{p}_i}{{\left| \Delta r \right|^2}} 
- \vec{p}_i \right)
\end{equation}
where $\epsilon_m$ is the dielectric constant of the medium and $\vec{\Delta r}
= \vec{r}_j - \vec{r}_i$ is a vector pointing from the dipole to the point where
the field is evaluated.

The polarizability of a sphere of radius $R$ is in the Rayleigh limit
\begin{equation}
 \alpha = 4 \pi \; \epsilon_0  \; R^3 \; \frac{\epsilon - \epsilon_m}{\epsilon +
2 \epsilon_m}
\end{equation}
We assume a Drude model with plasma frequency $\omega_P$ for the dielectric
properties of the sphere:
\begin{equation}
 \epsilon(\omega) = \epsilon_{\infty} - \frac{\omega_P}{ \omega \left(\omega \;
+ \; i\, \gamma \right) }
\end{equation}



By solving the equation system (\ref{eq:fields}) we now get the resonance
frequency $\omega$ of the coupled two-sphere system (assuming $\epsilon_m = 1$)
\begin{equation} \label{eq:omega_equal}
 \omega = \frac{\omega_P}{\sqrt{2 + \epsilon_{\infty}} }  \; \sqrt{ \frac{1 +
g}{ 1 +  \eta \; g}}
\end{equation}
with 
\begin{equation}
 \eta = \frac{\epsilon_{\infty} - 1}{\epsilon_{\infty} + 2 } 
 \qquad \text{and} \qquad
 g = n \; \frac{R^3}{d^3}  
\end{equation}
where $d = \left| \vec{r}_1 -  \vec{r}_2 \right|$ is the distance between the
centers of the spheres.
For the electric field being parallel to the pair axis, the index $n$ assumes
the value $-2$ for parallel dipoles (head to tail) and $2$ for anti-parallel
dipoles (head-to-head). When the electric field is perpendicular to the
pair-axis, $n$ is $+1$ for the parallel configuration and $-1$ for the
anti-parallel configuration.

One finds the anti-parallel solutions to the equation system (\ref{eq:fields})
by setting one dipole negative, e.g., setting $\vec{p}_2' = - \vec{p}_2$.


Up to here this is all in V.~Myroshnychenko et al.,\textit{Modeling the optical
response of gold nanoparticles}, Chem. Soc. rev. 2008, \textbf{37}, p.1972
chapter 5.2 and 9. By playing a bit with Mathematica we can expand this to
different sphere sizes and plasma frequencies. For the plasma frequencies we
have make the approximation of very small differences, but that is OK for our
purposes. With that we get
\begin{equation}\label{eq:omega_diff}
 \omega = \frac{\omega_{P,1} +  \omega_{P,2}}{2 \; \sqrt{2 + \epsilon_{\infty}}
}  \; \sqrt{ \frac{1 +
g}{ 1 +  \eta \; g}}
\end{equation}
with $\eta$ as above and
\begin{equation}
    g = n \; \frac{\left(R_1 \; R_2 \right)^{3/2}}{d^3}  
\end{equation}


\section{Interpretation}


We want to increase the  signal we get when the plasma frequency of one sphere
changes, i.e., the derivative with respect to $\omega_{P,2}$ should be larger
than what we would get without hybridization. For a single sphere we have
\begin{equation}
 \omega = \frac{\omega_P}{\sqrt{2 + \epsilon_{\infty}} } 
\end{equation}

$\eta$ is about $0.7$ for gold, as $\epsilon_{\infty} \approx 8$. The factor $g$
can be at most $1/4$ for touching spheres. From this follows that the
square-route term in eq. (\ref{eq:omega_equal}) and (\ref{eq:omega_diff}) is
close to one. The splitting of the lines by
hybridization is only some percent.


So hybridization does not increase the amount by witch the resonance is shifted
upon changing the plasma frequency of one sphere only. The factor $2$ in the
denominator of eq (\ref{eq:omega_diff})  with respect to eq
(\ref{eq:omega_equal}) even reduces the frequency shift. This is understandable
as we modify only part of the system,  in some cases even less than half of the
system's total volume.

However, we do not detect changes in resonance frequency, we measure changes in
transmission / absorption. The hybridized absorption peak is stronger than the
single sphere peak. It shifts less, but the product of both can be larger than
one.

Hybridizing two equal sized spheres produces a parallel ($n=-2$) peak of roughly
(CHECK) twice the  single sphere peak. It shifts by half the amount (eq.
(\ref{eq:omega_diff}) ) so the differential transmission signal is more or less
the same as a single sphere. Making the second sphere larger increases the
absorption peak, but also the anti-parallel peak ($n=+2$) comes up. However, it
should be possible to get a peak of more than twice the single sphere absorption
peak. Then we gain something.

Bottom line: Hybridization enhances the nonlinear signal of the plasma frequency
change of one sphere on the total transient absorption signal. This does not
happen by making the resonance position more strongly dependent on the plasma
frequency, but rather by suplying a lever (the strong absorption of the big
sphere) which is actuated less then when unhybridized.
 
 
 %----------
\section{Antenna enhancement in the plasmon hybridization picture}

The concept of plasmon hybridization allows to establish conditions for the
antenna enhancement. We demonstrate how the variation in a  plasmonic nanoparticle's dielectric properties has an increased influence on the extinction spectrum when coupling a second plasmonic particle with it.

\subsection{A single sphere}

\newcommand{\Evec}{{\bf E}}
\newcommand{\pvec}{{\bf p}}
\newcommand{\rvec}{{\bf r}}

We first discuss the reference case, i.e., a single plasmonic particle without
antenna. We approximate the particle by a sphere (radius $R$) in the Rayleigh limit. Its
polarizability $\alpha$ is given by
%
\begin{equation}
 \alpha = 4 \pi  \; R^3 \; \frac{\epsilon - \epsilon_m}{\epsilon +
2 \epsilon_m} \quad ,
\end{equation}
%
where $\epsilon$ and $\epsilon_m$ are the dielectric functions of the particle
and the medium, respectively.
We assume a Drude model with plasma frequency $\omega_P$ for the dielectric
properties of the sphere:
%
\begin{equation}
 \epsilon(\omega) = \epsilon_{\infty} - \frac{\omega_P}{ \omega \left(\omega \;
+ \; i\, \gamma \right) } \quad .
\end{equation}
%
The polarizability has a resonance at 
%
\begin{equation} 
 \omega_{\text{res}} = \frac{\omega_P}{\sqrt{2 + \epsilon_{\infty}} }  \quad ,
\end{equation}
%
assuming vacuum as medium ($\epsilon_m = 1$). The resonance is found as peak in
the extinction spectrum
%
\begin{equation} 
 C_{\text{ext}} = k \; \Im(\alpha) = \frac{k}{|\Evec^{\text{inc}}|^2} \; \Im(\Evec^{\text{inc}}\cdot  \bf{p}) \quad ,
\end{equation}
%
where $\Evec^{\text{inc}}$ is the incident electric field, $\bf{p} = \alpha \Evec^{\text{inc}}$ the induced
dipole moment, and $k = 2 \pi / \lambda $ the absolute value of the wave vector.

In the pump-probe experiment, a pump pulse modifies the dielectric properties of
the particle. In a first approximation we can assume that the plasma frequency
$\omega_P$ is shifted by the particle expansion to $\omega'_P = \omega_P (1 +
\delta)$ with $\delta \ll 1$. The new resonance position is then
%
\begin{equation} \label{eq:omega_equal}
 \omega_{\text{res'}} = \frac{\omega_P (1 + \delta)}{\sqrt{2 +
\epsilon_{\infty}} }  \quad .
\end{equation}

%----------------------------------------------------------
\section{Two coupled spheres}

Let us now turn to the case of the antenna enhancement. A second sphere of
different radius is brought close to the particle the signal of which we want to
increase. The first sphere is described by variables carrying the index 1; the
second antenna sphere has the index 2. The optical response of the spheres is
again described by an optical dipole. Each dipole experiences the incident field
$\bf{E}^{\text{inc}}(\bf{r}_i)$ and the field scattered from the other dipole.
The sum of these two fields multiplied by the dipoles polarizability $\alpha_i$
has to give in a self-consistent way the dipole moment (see, for example, ref. [Myroshnychenko])
%
\begin{equation} \label{eq:equationsystem}
     \pvec_1 = \alpha_1 \left( \Evec^{\text{inc}}(\rvec_1) +
\Evec^{\text{scat}}_2(\rvec_1) \right) \quad ,
\end{equation}
%
and vice versa. The scattered electrical near field $\Evec^{\text{scat}}$ of the dipole $i$ at position of the dipole $j$ is
%
\begin{equation}
  \Evec^{\text{scat}}_i(\rvec_j) = 
  \frac{1}{4 \pi \;  \epsilon_0 \; \epsilon_m} \; 
     \frac{1}{\left| \Delta \rvec \right|^3} 
          \left( \frac{3 \Delta \rvec \cdot \pvec_i}{\left| \Delta \rvec \right|^2}
- \pvec_i \right) \quad ,
\end{equation}
%
where $\Delta \rvec = \rvec_j - \rvec_i$ is a vector pointing from the dipole to the point where
the field is evaluated.



By solving the equation system \label{eq:equationsystem} we  get the resonance
frequency $\omega_{\text{res}} $ of the coupled two-sphere system (again assuming $\epsilon_m
= 1$)
%
\begin{equation}  \label{eq:omega_coupled}
 \omega_{\text{res}} = \frac{\omega_P}{\sqrt{2 + \epsilon_{\infty}} }  \; \sqrt{
\frac{1 + g}{ 1 +  \eta \; g}}
\end{equation}
%
with 
%
\begin{equation} \label{eq:omega_coupled_variables}
 \eta = \frac{\epsilon_{\infty} - 1}{\epsilon_{\infty} + 2 } 
 \qquad \text{and} \qquad
    g = n \; \frac{\left(R_1 \; R_2 \right)^{3/2}}{d^3}   \quad ,
\end{equation}
%
where $d = \left| \rvec_1 -  \rvec_2 \right|$ is the distance between the
centres of the spheres.
For the electric field being parallel to the pair axis, the index $n$ assumes
the value $-2$ for parallel dipoles (head to tail) and $2$ for anti-parallel
dipoles (head-to-head). When the electric field is perpendicular to the
pair-axis, $n$ is $+1$ for the parallel configuration and $-1$ for the
anti-parallel configuration.

In the pump-probe experiment, we are only interested in the influence of the
small particle's variation on the extinction spectrum. We therefore vary now the
plasma frequency of only one particle by $\delta \ll 1$ and calculate the new
resonance positions. We get
%
\begin{equation} \label{eq:omega_diff}
 \omega_{\text{res'}} = \frac{\omega_P (1 + \delta / 2)}{\sqrt{2 + \epsilon_{\infty}} }
 \; \sqrt{ \frac{1 + g}{ 1 +  \eta \; g}} \quad .
\end{equation}

Plasmon hybridization does not increase the amount by which the resonance is
shifted upon changing the plasma frequency of one sphere only. The shift is reduced by a factor of $2$ when comparing 
 eq. (\ref{eq:omega_diff})  with  eq.
(\ref{eq:omega_equal}). This can be understood
as we modify only part of the system,  in most cases even less than half of the
system's total volume.


However, the shift of the resonance position is only part of the answer to
signal enhancement, as we detect changes in transmission. Our signal is
proportional to the product of resonance shift and peak height of the extinction
resonance. A stronger extinction peak can overcompensate the reduced shift. For
the two-sphere system, the extinction spectrum is given by  
%
\begin{equation} 
 C_{ext} =  \frac{k}{|\Evec^{\text{inc}}|^2} \; \Im(\Evec_1 \cdot \pvec_1 + \Evec_2 \cdot \pvec_2) \quad ,
\end{equation}
%
where $\Evec_j$ is the electric field at dipole $j$. The spectrum shows to Lorentzian
peaks at the symmetric (+) and antisymmetric (-) resonances.  When varying the ratio of the sphere radius  $\zeta = R_2 / R_1$ while keeping the coupling factor $g$ in eqs. (\ref{eq:omega_coupled}) and (\ref{eq:omega_coupled_variables})
constant, we find the  amplitudes of
the two peaks  proportional to 
%
\begin{equation} 
 A_\pm = \frac{1}{2} \left( 1 \pm \zeta^{3/2} \right)^2 \quad .
\end{equation}
%
The sum of both amplitudes is, as expected, proportional to the total oscillator
strength
%
\begin{equation} 
 A_+ + A_- = 1 + \zeta^3 \quad .
\end{equation}
%
For two equal spheres ($\zeta = 1$), the symmetric mode caries twice the
oscillator strength of a single sphere ($A_+ = 2$) and the antisymmetric mode
is dark ($A_- = 0$). Already this twice as strong peak would compensate for the
reduction in peak shift calculated above. However, the antenna would not
enhance the signal. As soon as the second sphere becomes larger ($\zeta > 1$),
the symmetric mode continuously increases in amplitude and the antenna starts to
enhance the signal. In the dipole approximation we find  no upper bound for the
antenna enhancement.


\section*{References}


[Myroshnychenko] (CITE V.~Myroshnychenko
et al.,\textit{Modeling the optical response of gold nanoparticles}, Chem. Soc.
rev. 2008, \textbf{37}, p.1972 chapter 5.2 and 9)


