\documentclass[margin=5pt, tikz]{standalone}
\usepackage{pgfplots}
%\usepackage{xxcolor}
\pgfplotsset{compat=newest}

% Declare nice sphere shading: http://tex.stackexchange.com/a/54239/12440
\pgfdeclareradialshading[tikz@ball]{ball}{\pgfqpoint{0bp}{0bp}}{%
 color(0bp)=(tikz@ball!0!white);
 color(7bp)=(tikz@ball!0!white);
 color(15bp)=(tikz@ball!70!black);
 color(20bp)=(black!70);
 color(30bp)=(black!70)}
\makeatother

% Style to set TikZ camera angle, like PGFPlots `view`
\tikzset{viewport/.style 2 args={
    x={({cos(-#1)*1cm},{sin(-#1)*sin(#2)*1cm})},
    y={({-sin(-#1)*1cm},{cos(-#1)*sin(#2)*1cm})},
    z={(0,{cos(#2)*1cm})}
}}

% Styles to plot only points that are before or behind the sphere.
\pgfplotsset{only foreground/.style={
    restrict expr to domain={rawx*\CameraX + rawy*\CameraY + rawz*\CameraZ}{-0.05:100},
}}
\pgfplotsset{only background/.style={
    restrict expr to domain={rawx*\CameraX + rawy*\CameraY + rawz*\CameraZ}{-100:0.05}
}}

% Automatically plot transparent lines in background and solid lines in foreground
\def\addFGBGplot[#1]#2;{
    \addplot3[#1,only background, opacity=0.25] #2;
    \addplot3[#1,only foreground] #2;
}

% Automatically plot transparent lines in background and solid lines in foreground
\def\addFGBGplottable[#1]#2;{
    \addplot3 table[#1,only background, opacity=0.25] #2;
    \addplot3 table[#1,only foreground] #2;
}

\newcommand{\ViewAzimuth}{-30}
\newcommand{\ViewElevation}{30}

\begin{document}
\begin{tikzpicture}
%    % Compute camera unit vector for calculating depth
    \pgfmathsetmacro{\CameraX}{sin(\ViewAzimuth)*cos(\ViewElevation)}
    \pgfmathsetmacro{\CameraY}{-cos(\ViewAzimuth)*cos(\ViewElevation)}
    \pgfmathsetmacro{\CameraZ}{sin(\ViewElevation)}
    \path[use as bounding box] (-1,-1) rectangle (1,1); % Avoid jittering animation
    % Draw a nice looking sphere
    \begin{scope}
        \clip (0,0) circle (1);
        \begin{scope}[transform canvas={rotate=-20}]
            \shade [ball color=white] (0,0.5) ellipse (1.8 and 1.5);
        \end{scope}
    \end{scope}
    \begin{axis}[
        hide axis,
        view={\ViewAzimuth}{\ViewElevation},     % Set view angle
      %  every axis plot/.style={very thin},
        disabledatascaling,                      % Align PGFPlots coordinates with TikZ
        anchor=origin,                           % Align PGFPlots coordinates with TikZ
        viewport={\ViewAzimuth}{\ViewElevation}, % Align PGFPlots coordinates with TikZ
    ]   
    \addplot3 table[x index=0, y index=1, z index=2,only marks,scatter,mark=cube* ] {rabi075.dat};
        % Plot equator and two longitude lines with occlusion
       \addFGBGplot[domain=0:2*pi, samples=100, samples y=1] ({cos(deg(x))}, {sin(deg(x))}, 0);
        \addFGBGplot[domain=0:2*pi, samples=100, samples y=1] (0, {sin(deg(x))}, {cos(deg(x))});
        \addFGBGplot[domain=0:2*pi, samples=100, samples y=1] ({sin(deg(x))}, 0, {cos(deg(x))});
      \addFGBGplot[domain=0:2*pi, samples=100, samples y=1] ({sin(deg(x))}, 0, {cos(deg(x))});
 
    \end{axis}
\end{tikzpicture}
\end{document}