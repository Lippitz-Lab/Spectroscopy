

\chapter{(Perturbed) Free Induction Decay}



\section{Tasks}

\begin{itemize}
\item Simulated CW / CD papers
\end{itemize}




\section{Experiment}

some text

\section{Coherence as a Source of Radiation}


Let us look at methods to measure elements of the density matrix $\rho$. We can measure populations, i.e., diagonal elements of $\rho$ by fluorescence emission or electron tunneling. If an atom, molecule, quantum dot is in the excited state, it can emit a fluorescence photon and revert to the ground state. All coherence is lost in this process, neither the fluorescence photon nor the ground state carry any phase relation to the excited state. The excite state is also destroyed, as afterwards the system is in the ground state. But we can observe the fluorescence photon and from the fluorescence rate we can determine how many systems of an ensemble or how often a single system is (better: was) in the excited state. We thus measure population of the emitting state. In the same way, we can use electrons tunneling out of the excited state, for example in a diode structure which also supplies  a new electron to the ground state. Also this tunneling signal is incoherent.

We can also measure coherences, i.e., off-diagonal elements in the density matrix $\rho$, as these coherences are the source of radiation. To see this, we need to connect the microscopic description by the density matrix to the macroscopic world of Maxwell's equations, resulting in the Maxwell-Bloch equations \footcite[chapter 8.3]{MilonniEberly1988} \footcite[chapter 3.9]{Rand2016}\footcite{Meschede-OLL}. This is what the expectation value does. The  polarization $p$ of a single two-level system at position $z$ in the pumping laser beam is given by the expectation value of the polarization operator $\hat{\mu}$
\[
 p(t,z) = \braket{\hat{\mu}} = Tr ( \mu \, \rho) = \mu_{01} \rho_{10} + c.c. =   \mu_{01}  \rho_{10}' \, e^{-i (\omega_L t - k z)}  + c.c.
\]
where the polarization operator has only off-diagonal entries in the matrix representation. The prime signals once more the density matrix in the rotating frame. The macroscopic polarisation $P = N \, p$ of a volume of identical atoms is a source term in the one-dimensional wave equation
\[
  \frac{\partial^2}{\partial z^2} \boldsymbol{E}_S  - \frac{1}{c^2} \frac{\partial^2}{\partial t^2} \boldsymbol{E}_S 
 =  
\frac{1}{c^2\, \epsilon_0} \frac{\partial^2}{\partial t^2} \boldsymbol{P}  
\]
$E_S$ is the (slowly varying) amplitude of the generated field
\[
 \boldsymbol{E}_S = \frac{1}{2} \, E_s(z,t) \, e^{-i (\omega_L t - k z)}  + c.c. \quad .
\]
Also in the wave equation we  use of the slowly-varying envelope approximation and get (using $\rho_{10}' = u - i v$)
\[
  \frac{\partial}{\partial z} E_S  - \frac{1}{c} \frac{\partial}{\partial t} E_S
 =  
N \frac{i k }{2 \epsilon_0}  \mu_{01} ( u - i v)
\]
This forms together with the Bloch equations from last chapter the Maxwell-Bloch equations of a coupled light-matter system. As solution we find
\[
 E_S = N L \frac{i k }{2 \epsilon_0}  \mu_{01} ( u - i v)
 =
  N L \frac{k }{2 \epsilon_0}  \mu_{01} (v + i u)
  \propto \mu_{01} \Im (\rho_{01}' ) 
\]
It is thus the $v$ component of the Bloch vector (or $\Im (\rho_{01}'$) ) that produces the optical field\footcite[chapter V.B.1]{CT-atom-photon}, as the $ + c.c.$  in the definition of $ \boldsymbol{E}_S $ selects the real part of $ E_S$.
%
%The resulting field is\footcite[eq. 4.4]{Hamm-dummies}\footcite[eq. 3.9.16 !! update nomenclature !!]{Rand2016}
%\[
%  E_{out} \propto -i \, P \propto - \mu_{01} \, \Im ( \rho_{01} )
%\]


\section{Absorption of a single photon}

Let us discuss as example the absorption of a single photon, which transfers the system from the ground state to the excited state, or equivalently is the action of a $\pi$-pulse. We start by a two-level system in den ground state. The Bloch vector points to the north pole. We shine in an optical field on resonance with the system ($\omega_0 = \omega_L$). The duration of the light pulse $\Delta t$ should be such that is a $\pi$-pulse, i.e.
\[
 \theta = \pi = \int_0^{\Delta t} \, \Omega(t) \, dt =  \int_0^{\Delta t} \, \frac{2 \, \mu \, E(t)}{\hbar} \, dt
\]
When the light pulse has the amplitude $E_0$ during $t= 0 \cdots \Delta t$ we get
\[
 \Delta t = \pi \frac{\hbar}{2 \mu E_0}  = \pi \, \frac{1}{\Omega}
\] 
The $w$ component of the Bloch vector follows the Rabi oscillation, i.e.
\[
 w(t) = \cos ( \Omega t  ) = \cos \left( \pi  \frac{t}{\Delta t} \right)
\]
and the excited state population accordingly
\[
 \rho'_{11}(t) = \frac{1 - w}{2}  = \sin^2 ( \Omega t /2 )
\]
The imaginary part of the coherence, related to the $v$ component of the Bloch vector, is
\[
 \Im (\rho'_{01} ) = \frac{1}{2} v =  - \frac{1}{2}  \sin \left( \pi \frac{t}{\Delta t} \right)
\]
The radiated field is 
\[
 E_S  = N L \frac{k }{\epsilon_0}  \mu_{01} \Im (\rho_{01}' ) 
 = - N L \frac{k }{2 \epsilon_0}  \mu_{01} \sin \left( \pi \frac{t}{\Delta t} \right)
\]
On the detector, the pumping field $E_0$ and the radiated field $E_S$ interfere. We measure the power $P$ of a single pulse per area of the beam  
\[
P  = \frac{1}{2} \epsilon_0 c \, \int_\text{pulse} | E_0 + E_S |^2 \ dt
\]
The presence of the absorbing two-level system leads to a change in detected power density, assuming $E_S \ll E_0$
\[
 \Delta P = \frac{1}{2} \epsilon_0 c \,  
\int_\text{pulse} 2 E_0 E_S  \ dt
  =-   \sigma \frac{k c }{2}  E_0 \mu_{01}  \int_0^{\Delta t}   \sin \left( \pi \frac{t}{\Delta t} \right)  \,  dt
\]
where $\sigma = N L $ is the projected area density of the two-level systems. The integral gives $2 \Delta t/\pi$ so that
\[
 \Delta P =-  \sigma \frac{k c}{2}  E_0 \mu_{01} \frac{2 \Delta t}{\pi}
= -  \sigma \frac{k c}{2}  E_0 \mu_{01}  \frac{\hbar}{\mu E_0} 
= -  \sigma \, \frac{ \hbar \omega }{2 }  
\]
Each atom removes the energy of one photon, decorated by the factor $1/2$ which should not be there ....

\printbibliography[segment=\therefsegment,heading=subbibliography]
