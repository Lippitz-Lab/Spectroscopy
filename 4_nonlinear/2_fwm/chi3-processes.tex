\documentclass[margin=0mm]{standalone}
\usepackage{tikz}
\usepackage{pgfplots}
 \pgfplotsset{compat=newest}


\usepackage{amssymb,amsmath}

\usepackage[T1]{fontenc}
\usepackage[utf8]{inputenc}



%

%% -- tint overrides
%% fonts, using roboto (condensed) as default
\usepackage[sfdefault,condensed]{roboto}
%% also nice: \usepackage[default]{lato}

%% colored links, setting 'borrowed' from RJournal.sty with 'Thanks, Achim!'
\RequirePackage{color}
\definecolor{link}{rgb}{0.1,0.1,0.8} %% blue with some grey
\hypersetup{
  colorlinks,%
  citecolor=link,%
  filecolor=link,%
  linkcolor=link,%
  urlcolor=link
}

%% macros
\makeatletter

%% -- tint does not use italics or allcaps in title
\renewcommand{\maketitle}{%     
  \newpage
  \global\@topnum\z@% prevent floats from being placed at the top of the page
  \begingroup
    \setlength{\parindent}{0pt}%
    \setlength{\parskip}{4pt}%
    \let\@@title\@empty
    \let\@@author\@empty
    \let\@@date\@empty
    \ifthenelse{\boolean{@tufte@sfsidenotes}}{%
      %\gdef\@@title{\sffamily\LARGE\allcaps{\@title}\par}%
      %\gdef\@@author{\sffamily\Large\allcaps{\@author}\par}%
      %\gdef\@@date{\sffamily\Large\allcaps{\@date}\par}%
      \gdef\@@title{\begingroup\fontseries{b}\selectfont\LARGE{\@title}\par}%
      \gdef\@@author{\begingroup\fontseries{l}\selectfont\Large{\@author}\par}%
      \gdef\@@date{\begingroup\fontseries{l}\selectfont\Large{\@date}\par}%
    }{%
      %\gdef\@@title{\LARGE\itshape\@title\par}%
      %\gdef\@@author{\Large\itshape\@author\par}%
      %\gdef\@@date{\Large\itshape\@date\par}%
      %\gdef\@@title{\begingroup\fontseries{b}\selectfont\LARGE\@title\par\endgroup}%
      %\gdef\@@author{\begingroup\fontseries{l}\selectfont\Large\@author\par\endgroup}%
      %\gdef\@@date{\begingroup\fontseries{l}\selectfont\Large\@date\par\endgroup}%
      \gdef\@@title{\begingroup\fontseries{b}\fontsize{28}{60}\selectfont\@title\par\endgroup}%
      \gdef\@@author{\begingroup\fontseries{l}\fontsize{16}{20}\selectfont\@author\par\endgroup}%
      \gdef\@@date{\begingroup\fontseries{l}\fontsize{16}{20}\selectfont\@date\par\endgroup}%
    }%
    %\phantom{XXX}%
    \vspace{12pc}%
    \@@title%
    \vspace{4pc}%
    \@@author
    \@@date
  \endgroup
  \thispagestyle{plain}% suppress the running head
  \tuftebreak% add some space before the text begins
  \@afterindentfalse\@afterheading% suppress indentation of the next paragraph
}

%% -- tint does not use italics or allcaps in section/subsection/paragraph
\titleformat{\chapter}%
  [display]% shape
  {\relax\ifthenelse{\NOT\boolean{@tufte@symmetric}}{\begin{fullwidth}}{}}% format applied to label+text
  %{\itshape\huge\thechapter}% label
  {\huge Chapter \thechapter}% label
  {0pt}% horizontal separation between label and title body
  %{\huge\rmfamily\itshape}% before the title body
  {\fontseries{b}\selectfont\huge}% before the title body
  [\ifthenelse{\NOT\boolean{@tufte@symmetric}}{\end{fullwidth}}{}]% after the title body

\titleformat{\section}%
  [hang]% shape
  %{\normalfont\Large\itshape}% format applied to label+text
  {\fontseries{b}\selectfont\Large}% format applied to label+text
  {\thesection}% label
  {1em}% horizontal separation between label and title body
  {}% before the title body
  []% after the title body

\titleformat{\subsection}%
  [hang]% shape
  %{\normalfont\large\itshape}% format applied to label+text
  {\fontseries{m}\selectfont\large}% format applied to label+text
  {\thesubsection}% label
  {1em}% horizontal separation between label and title body
  {}% before the title body
  []% after the title body

\titleformat{\paragraph}%
  [runin]% shape
  %{\normalfont\itshape}% format applied to label+text
  {\fontseries{l}\selectfont}% format applied to label+text
  {\theparagraph}% label
  {1em}% horizontal separation between label and title body
  {}% before the title body
  []% after the title body

%% -- tint does not use italics here either
% Formatting for main TOC (printed in front matter)
% {section} [left] {above} {before w/label} {before w/o label} {filler + page} [after]
\ifthenelse{\boolean{@tufte@toc}}{%
  \titlecontents{part}% FIXME
    [0em] % distance from left margin
    %{\vspace{1.5\baselineskip}\begin{fullwidth}\LARGE\rmfamily\itshape} % above (global formatting of entry)
    {\vspace{1.5\baselineskip}\begin{fullwidth}\fontseries{m}\selectfont\LARGE} % above (global formatting of entry)
    {\contentslabel{2em}} % before w/label (label = ``II'')
    {} % before w/o label
    {\rmfamily\upshape\qquad\thecontentspage} % filler + page (leaders and page num)
    [\end{fullwidth}] % after
  \titlecontents{chapter}%
    [0em] % distance from left margin
    %{\vspace{1.5\baselineskip}\begin{fullwidth}\LARGE\rmfamily\itshape} % above (global formatting of entry)
    {\vspace{1.5\baselineskip}\begin{fullwidth}\fontseries{m}\selectfont\LARGE} % above (global formatting of entry)
    {\hspace*{0em}\contentslabel{2em}} % before w/label (label = ``2'')
    {\hspace*{0em}} % before w/o label
    %{\rmfamily\upshape\qquad\thecontentspage} % filler + page (leaders and page num)
    {\upshape\qquad\thecontentspage} % filler + page (leaders and page num)
    [\end{fullwidth}] % after
  \titlecontents{section}% FIXME
    [0em] % distance from left margin
    %{\vspace{0\baselineskip}\begin{fullwidth}\Large\rmfamily\itshape} % above (global formatting of entry)
    {\vspace{0\baselineskip}\begin{fullwidth}\fontseries{m}\selectfont\Large} % above (global formatting of entry)
    {\hspace*{2em}\contentslabel{2em}} % before w/label (label = ``2.6'')
    {\hspace*{2em}} % before w/o label
    %{\rmfamily\upshape\qquad\thecontentspage} % filler + page (leaders and page num)
    {\upshape\qquad\thecontentspage} % filler + page (leaders and page num)
    [\end{fullwidth}] % after
  \titlecontents{subsection}% FIXME
    [0em] % distance from left margin
    %{\vspace{0\baselineskip}\begin{fullwidth}\large\rmfamily\itshape} % above (global formatting of entry)
    {\vspace{0\baselineskip}\begin{fullwidth}\fontseries{m}\selectfont\large} % above (global formatting of entry)
    {\hspace*{4em}\contentslabel{4em}} % before w/label (label = ``2.6.1'')
    {\hspace*{4em}} % before w/o label
    %{\rmfamily\upshape\qquad\thecontentspage} % filler + page (leaders and page num)
    {\upshape\qquad\thecontentspage} % filler + page (leaders and page num)
    [\end{fullwidth}] % after
  \titlecontents{paragraph}% FIXME
    [0em] % distance from left margin
    %{\vspace{0\baselineskip}\begin{fullwidth}\normalsize\rmfamily\itshape} % above (global formatting of entry)
    {\vspace{0\baselineskip}\begin{fullwidth}\fontseries{m}\selectfont\normalsize\rmfamily} % above (global formatting of entry)
    {\hspace*{6em}\contentslabel{2em}} % before w/label (label = ``2.6.0.0.1'')
    {\hspace*{6em}} % before w/o label
    %{\rmfamily\upshape\qquad\thecontentspage} % filler + page (leaders and page num)
    {\upshape\qquad\thecontentspage} % filler + page (leaders and page num)
    [\end{fullwidth}] % after
}{}

% tint: no smallcaps in header 
% The 'fancy' page style is the default style for all pages.
\fancyhf{} % clear header and footer fields
\ifthenelse{\boolean{@tufte@twoside}}
  %{\fancyhead[LE]{\thepage\quad\smallcaps{\newlinetospace{\plainauthor}}}%
  %  \fancyhead[RO]{\smallcaps{\newlinetospace{\plaintitle}}\quad\thepage}}
  %{\fancyhead[RE,RO]{\smallcaps{\newlinetospace{\plaintitle}}\quad\thepage}}
  {\fancyhead[LE]{\thepage\quad{\newlinetospace{\plaintitle}}}%
    \fancyhead[RO]{{\newlinetospace{\plaintitle}}\quad\thepage}}%
  {\fancyhead[RE,RO]{{\newlinetospace{\plaintitle}}\quad\thepage}}
  



\makeatother




\begin{document}

\usetikzlibrary{math} %needed tikz library
\usetikzlibrary{arrows.meta} %needed tikz library

\tikzset{>=Latex}

\begin{tikzpicture}
\useasboundingbox (0,0.5) rectangle (10.6,4.1);
%\draw (0,0.5) rectangle (10.6,4.1);

\tikzmath{ \width =0.6; \gs = 1; \xoffs=0.1;  \gap = 0.55;}

% THG ---------------
\tikzmath{\center = \gap + \width; \es=4;}

\draw (\center - \width ,\gs) -- ++ ( 2 *\width,0)  node[midway,below] {THG}; 
\draw[dashed] (\center - \width ,\es) -- ++ ( 2 *\width,0);

\tikzmath{\thglength = (\es - \gs) / 3; };

\draw[ ->] (\center - \xoffs, \gs) -- ++ (0, \thglength )  node[midway, rotate=90, above] {$\omega$}; 
\draw[ ->] (\center - \xoffs, \gs  + \thglength) -- ++ (0, \thglength )  node[midway, rotate=90, above] {$\omega$}; 
\draw[ ->] (\center - \xoffs, \gs + 2 *\thglength) -- ++ (0, \thglength )  node[midway, rotate=90, above] {$\omega$}; 

\draw[ ->] (\center + \xoffs, \es) -- ++ (0, -3 * \thglength )  node[midway, rotate=90, below] {$3 \,\omega$}; 

 % PP -------------
 \tikzmath{\rightend = \center + \width;}
\tikzmath{\width=1.2; \es1=3.5; \es2 = 4; \offspp = 0.6;}
\tikzmath{\center = \rightend + \gap + \width;}

\draw (\center - \width ,\gs) -- ++ ( 2 *\width,0)  node[midway,below] {pump-probe}; 
\draw (\center - \width ,\es1) -- ++ ( 2 *\width,0);
\draw (\center - \width ,\es2) -- ++ ( 2 *\width,0);

\draw[ ->] (\center - \xoffs - \offspp, \gs) -- ++ (0, \es1 - \gs )  node[midway, rotate=90, above] {$\omega_\text{pump}$}; 
\draw[ ->] (\center + \xoffs - \offspp, \es1) -- ++ (0, \gs - \es1 )  node[midway, rotate=90, below] {$\omega_\text{pump}$}; 

\draw[ ->] (\center - \xoffs + \offspp, \gs) -- ++ (0, \es2 - \gs )  node[midway, rotate=90, above] {$\omega_\text{probe}$}; 
\draw[ ->] (\center + \xoffs + \offspp, \es2) -- ++ (0, \gs - \es2 )  node[midway, rotate=90, below] {$\omega_\text{probe}$}; 

\draw[ <->]  (\center + \xoffs - \offspp + 0.2, \gs + 0.2) --(\center - \xoffs + \offspp - 0.2, \gs + 0.2) node[midway, above] {$\Delta t$}; 


% CARS --------------
\tikzmath{\rightend = \center + \width;}
\tikzmath{\width=1.2; \es1=3.5; \es2 = 4; \vib = 1.5; \offspp = 0.6;}
\tikzmath{\center = \rightend + \gap + \width;}


\draw (\center - \width ,\gs) -- ++ ( 2 *\width,0)  node[midway,below] {CARS}; 
\draw (\center - \width ,\vib) -- ++ ( 2 *\width,0);
\draw[dashed] (\center - \width ,\es1) -- ++ ( 2 *\width,0);
\draw [dashed](\center - \width ,\es2) -- ++ ( 2 *\width,0);

\draw[ ->] (\center - \xoffs - \offspp, \gs) -- ++ (0, \es1 - \gs )  node[midway, rotate=90, above] {$\omega_\text{pump}$}; 
\draw[ ->] (\center + \xoffs - \offspp, \es1) -- ++ (0, \vib - \es1 )  node[midway, rotate=90, below] {$\omega_\text{Stokes}$}; 

\draw[ ->] (\center - \xoffs + \offspp, \vib) -- ++ (0, \es2 - \vib )  node[midway, rotate=90, above] {$\omega_\text{pump}$}; 
\draw[ ->] (\center + \xoffs + \offspp, \es2) -- ++ (0, \gs - \es2 )  node[midway, rotate=90, below] {$\omega_\text{CARS}$}; 

\draw [ |-| ] (\center + \width + 0.2, \gs) -- ++ (0, \vib - \gs) node[midway,right] {$\Omega_\text{vib}$};
 
 
 % FWM --------------
 \tikzmath{\rightend = \center + \width + 0.7;}
\tikzmath{ \width=0.6; \es=4; \l1 = 2; \l3 = 1.8;}
\tikzmath{\center = \rightend + \gap + \width;}


\draw (\center - \width ,\gs) -- ++ ( 2 *\width,0)  node[midway,below] {FWM}; 
\draw[dashed] (\center - \width ,\es) -- ++ ( 2 *\width,0);

\tikzmath{\l2 = (\es - \gs)  - \l1; };
\tikzmath{\l4 = (\es - \gs)  - \l3; };

\draw[ ->] (\center - \xoffs, \gs) -- ++ (0, \l1 )  node[midway, rotate=90, above] {$\omega_1$}; 
\draw[ ->] (\center - \xoffs , \gs  + \l1) -- ++ (0, \l2 )  node[midway, rotate=90, above] {$\omega_2$}; 


\draw[ ->] (\center + \xoffs , \es) -- ++ (0, -1 * \l3 )  node[midway, rotate=90, below] {$\omega_3$}; 
\draw[ <-] (\center + \xoffs, \gs) -- ++ (0,  \l4 )  node[midway, rotate=90, below] {$\omega_4$}; 

\end{tikzpicture}


\end{document}