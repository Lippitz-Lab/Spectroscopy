
\documentclass[margin=0mm]{standalone}
\usepackage[utf8]{inputenc}
\usepackage{tikz}

\begin{document}

\usetikzlibrary{math,matrix,fit,positioning}

\usetikzlibrary{calc}
\usetikzlibrary{arrows.meta} %needed tikz library

\tikzset{>=latex}

\tikzset{
mymat/.style={
    matrix of math nodes,
    left delimiter=|, right delimiter=|,
    align=center,
    column sep=-\pgflinewidth,
}
%,mymats/.style={
%    mymat,
%    nodes={draw,fill=#1}
%} 
 }
 
\newcommand{\myarrow}[5]{\draw[#4](#1.south -| #2)  -- ++(#3 :6mm) node[above,pos=0.55]{$#5$};
} 

\newcommand{\interactLp}[3]{\myarrow{#1-#2-1}{#1.west}{-135}{<-}{#3}} 
\newcommand{\interactLm}[3]{\myarrow{#1-#2-1}{#1.west}{+135}{->}{#3}} 
\newcommand{\interactRp}[3]{\myarrow{#1-#2-2}{#1.east}{ -45}{<-}{#3}} 
\newcommand{\interactRm}[3]{\myarrow{#1-#2-2}{#1.east}{ +45}{->}{#3}}  

\newcommand{\interactout}[2]{\myarrow{#1-1-1}{#1.west}{+135}{->,dashed}{#2}} 


\begin{tikzpicture}[
     every left delimiter/.style={xshift=.4em},
     every right delimiter/.style={xshift=-.4em},
]
\useasboundingbox (0.,0) rectangle (10.6,5.1);

\tikzmath{ \y1 = 0.8; \x1 = 1.3; \dx = 3; \y2 = 3.5;}

%\draw (0.,0) rectangle (10.6,5.1);

%-------------
\matrix[mymat] at (\x1,\y1) (mat1)
{   1 & 1\\
    1 & 0\\
    0 & 0\\
};

\interactRp{mat1}{1}{pu};
\interactLp{mat1}{2}{pu};


%-------------
\matrix[mymat] at (\x1 + 1 * \dx,\y1) (mat2)
{   1 & 1\\
    0 & 1\\
    0 & 0\\
};

\interactLp{mat2}{1}{pu};
\interactRp{mat2}{2}{pu};


%-------------
\matrix[mymat] at (\x1 + 2 * \dx,\y1) (mat3)
{   0 & 0\\
    1 & 0\\
    0 & 0\\
};

\interactLm{mat3}{1}{pu};
\interactLp{mat3}{2}{pu};



%-------------
\matrix[mymat] at (\x1 + 3 * \dx -1,\y1) (mat4)
{   0 & 0\\
    0 & 1\\
    0 & 0\\
};

\interactRm{mat4}{1}{pu};
\interactRp{mat4}{2}{pu};

%-------------

\matrix[mymat] at (\x1 + 0 * \dx,\y2) (mata)
{   a & a \\
    b & a \\
    b & b \\
};



\interactout{mata}{out};
\interactRm{mata}{2}{pr};

%-------------

\matrix[mymat] at (\x1 + 1 * \dx,\y2) (matb) 
{   a & a \\
    b & a \\
    a & a \\
} ;



\interactout{matb}{out};
\interactLp{matb}{2}{pr};

\node[draw,yshift=5mm] at (mata.north) {a};
\node[draw,yshift=5mm] at (matb.north) {b};

\node[draw,yshift=5mm] at (mat1.north) {1};
\node[draw,yshift=5mm] at (mat2.north) {2};
\node[draw,yshift=5mm] at (mat3.north) {3};
\node[draw,yshift=5mm] at (mat4.north) {4};
\end{tikzpicture}
\end{document}
