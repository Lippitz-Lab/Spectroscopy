%\documentclass[margin=0mm]{standalone}
%\usepackage{tikz}
%\usepackage{pgfplots}
% \pgfplotsset{compat=newest}
%
%
%\usepgfplotslibrary{groupplots}
%
%\usepackage{currfile,hyperxmp,calc}
%\usetikzlibrary{calc,math}
%\usetikzlibrary{calc,patterns,angles,quotes}
%
%
%\usetikzlibrary{angles,quotes}
%
%\usetikzlibrary{math,matrix,fit,positioning,intersections}
%\usetikzlibrary{patterns,decorations.pathmorphing,decorations.markings}
%\usetikzlibrary{calc}
%\usetikzlibrary{arrows.meta} %needed tikz library
%\usetikzlibrary{quotes,angles}
%
%\usepackage{braket}
%
% 
%
%
%
%
%\begin{document}


\begin{tikzpicture}
%\useasboundingbox (0.,0) rectangle (10.6,8.3);

\tikzmath{ \dx1 = -0.4; \dx2=-0.2; \dx3=+0.3;}

%\draw (0.,0) rectangle (10.6,8.3);

%-------------
\matrix[ matrix of math nodes, column sep = 4mm, row sep = 3mm]  (mat)
{   \ket{0}\bra{0} & \ket{1}\bra{0} & \ket{2}\bra{0} \\
    \ket{0}\bra{1} & \ket{1}\bra{1} & \ket{2}\bra{1} \\
  \ket{0}\bra{2} & \ket{1}\bra{2} & \ket{2}\bra{2}  \\
  };


\draw[->, blue] ($(mat-1-1.south) + (\dx1,0)$) --  ($(mat-2-1.north) + (\dx1,0)$);
\draw[->, blue] ($(mat-2-1.south) + (\dx1,0)$) --  ($(mat-3-1.north) + (\dx1,0)$);

\draw[<-, blue] ($(mat-1-1.south) + (\dx2,0)$) --  ($(mat-2-1.north) + (\dx2,0)$);
\draw[<-, blue] ($(mat-2-1.south) + (\dx2,0)$) --  ($(mat-3-1.north) + (\dx2,0)$);

\draw[->, red] (mat-1-1) --(mat-1-2); 
\draw[->, red] (mat-1-2) --(mat-2-2); 
\draw[->, red] (mat-2-2) --(mat-2-1); 
\draw[<-, red] ($(mat-1-1.south) + (\dx3,0)$) --  ($(mat-2-1.north) + (\dx3,0)$);


\end{tikzpicture}



%\end{document}