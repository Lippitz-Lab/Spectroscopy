\documentclass[margin=0mm]{standalone}
\usepackage{tikz}
\usepackage{pgfplots}
 \pgfplotsset{compat=newest}


\usepgfplotslibrary{groupplots}

\usepackage{currfile,hyperxmp,calc}
\usetikzlibrary{calc,math}
\usetikzlibrary{calc,patterns,angles,quotes}


\usetikzlibrary{angles,quotes}

\usetikzlibrary{math,matrix,fit,positioning,intersections}
\usetikzlibrary{patterns,decorations.pathmorphing,decorations.markings}
\usetikzlibrary{calc}
\usetikzlibrary{arrows.meta} %needed tikz library
\usetikzlibrary{quotes,angles}

\usepackage{braket}

 




\tikzset{
mymat/.style={
    matrix of math nodes,
    left delimiter=|, right delimiter=|,
    align=center,
    column sep=-\pgflinewidth,
}
}


\begin{document}

\newcommand{\myarrow}[5]{\draw[#4](#1.south -| #2)  -- ++(#3 :6mm) node[above,pos=1]{$#5$};
} 

\newcommand{\interactLp}[3]{\myarrow{#1-#2-1}{#1.west}{-135}{<-}{#3}} 
\newcommand{\interactLm}[3]{\myarrow{#1-#2-1}{#1.west}{+135}{->}{#3}} 
\newcommand{\interactRp}[3]{\myarrow{#1-#2-2}{#1.east}{ -45}{<-}{#3}} 
\newcommand{\interactRm}[3]{\myarrow{#1-#2-2}{#1.east}{ +45}{->}{#3}}  

\newcommand{\interactout}[2]{\myarrow{#1-1-1}{#1.west}{+135}{->,dashed}{#2}} 

\begin{tikzpicture}[
     every left delimiter/.style={xshift=0.6em},
     every right delimiter/.style={xshift=-0.6em} ,  ]
%\useasboundingbox (0.,0) rectangle (10.6,8.3);

%\tikzmath{ \y1 = 1.8; \x1 = 1.1; \dx = 2.7; \y2 = 6;}

%\draw (0.,0) rectangle (10.6,8.3);

%-------------
\matrix[mymat]  (mat1)
{   0 & 0\\
    1 & 0\\
    1 & 1\\
     1 & 0\\
    0 & 0\\
};

\interactout{mat1}{out};
\interactLp{mat1}{4}{k_1};
\interactRp{mat1}{3}{k_2};
\interactRm{mat1}{2}{k_3};

\draw[dashed, blue] (mat1-1-2.south east) -- (mat1-1-1.south west);
\draw[dashed, blue] (mat1-2-2.south east) -- (mat1-2-1.south west);
\draw[dashed, blue] (mat1-3-2.south east) -- (mat1-3-1.south west);
\draw[dashed, blue] (mat1-4-2.south east) -- (mat1-4-1.south west);

\draw[->, blue] (mat1-5-1.center) .. controls (mat1-5-1.north east) .. (mat1-4-1.center) ;
\draw[->, blue] (mat1-4-2.center) .. controls (mat1-4-2.north west) .. (mat1-3-2.center) ;
\draw[->, blue] (mat1-3-2.center) .. controls (mat1-3-2.north west) .. (mat1-2-2.center) ;
\draw[->, blue] (mat1-2-1.center) .. controls (mat1-2-1.north east) .. (mat1-1-1.center) ;

\node [blue, yshift=-7mm,xshift=-1mm] at  (mat1-5-1.center) {$\ket{n}$};
\node [blue, yshift=-7mm] at  (mat1-5-2.center) {$\bra{m}$};

\draw [->, blue] ($(mat1-5-1.south)  - (14mm,0)$) --  node[rotate=90,above] {time} ($(mat1-1-1.north)  - (14mm,0)$);

\end{tikzpicture}



\end{document}