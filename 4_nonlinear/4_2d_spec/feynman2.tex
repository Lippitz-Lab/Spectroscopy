
\documentclass[margin=0mm]{standalone}
\usepackage[utf8]{inputenc}
\usepackage{tikz}

\begin{document}

\usetikzlibrary{matrix,fit,positioning}

\usetikzlibrary{calc}
\usetikzlibrary{arrows.meta} %needed tikz library

\tikzset{>=latex}

\tikzset{
mymat/.style={
    matrix of math nodes,
    left delimiter=|, right delimiter=|,
    align=center,
    column sep=-\pgflinewidth,
}
%,mymats/.style={
%    mymat,
%    nodes={draw,fill=#1}
%} 
 }
 
\newcommand{\myarrow}[5]{\draw[#4](#1.south -| mat1.#2)  -- ++(#3 :6mm) node[above,pos=0.55]{$#5$};
} 

\newcommand{\interactLp}[2]{\myarrow{#1-1}{west}{-135}{<-}{#2}} 
\newcommand{\interactLm}[2]{\myarrow{#1-1}{west}{+135}{->}{#2}} 
\newcommand{\interactRp}[2]{\myarrow{#1-2}{east}{ -45}{<-}{#2}} 
\newcommand{\interactRm}[2]{\myarrow{#1-2}{east}{ +45}{->}{#2}}  
\newcommand{\interactout}[2]{\myarrow{#1-1-1}{west}{+135}{->,dashed}{#2}} 


\begin{tikzpicture}[
     every left delimiter/.style={xshift=.4em},
     every right delimiter/.style={xshift=-.4em},
]
%\useasboundingbox (0.,0) rectangle (6.0,3.2);

%\draw (0.,0) rectangle (6.0,3.2);


\matrix[mymat] at (0,0) (mat1)
{   0 &  0\\
    1 & 0\\
    0 & 0\\
    0 & 1\\
    0 & 0\\
};

\interactout{mat1}{4};

\interactRp{mat1-4}{1};
\interactRm{mat1-3}{2};

\interactLp{mat1-2}{3};

%\interactLm{mat1-2}{2};
%\interactLp{mat1-2}{3};
%\interactRp{mat1-2}{4};

%\myarrow[+]{2};

\end{tikzpicture}
\end{document}
