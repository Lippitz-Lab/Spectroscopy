\renewcommand{\lastmod}{May 7, 2020}
\chapter{Pulse shaping}


\section{Tasks}

\begin{itemize}
\item lkdlkweq
\end{itemize}

\section{Experiment}



\section{Generation of short pulses}


All non-linear effects  require high
intensities of at least one wave. This 
is relatively
easy to achieve, if not one for the entire period
constantly high intensity is required, but periodic
intensity peaks are sufficient. During these pulses the
intensity  is sufficiently high for nonlinear optics. Between the pulses 
the intensity is so low that no other processes take place. The averaged power of such a laser can therefore
be in the range of about one Watt and still 
peak values of about 100~kW are reached when a pulse duration of
100~fs is compared to a pulse spacing of 13~ns. In
this section we will discuss how such short
laser pulses can be generated.


The starting point is the  wide gain spectrum of a
titanium-sapphire laser. The coupling of the atomic titanium levels to
vibration states of the sapphire crystal cause a strong
homogeneous widening of the transitions\footcite[chapter 4]{Rulliere2005}, so that
this type of laser can operate in the wavelength range 700--1000~nm. This allows many longitudinal modes $E_n$  of
electric field in the cavity  to reach the laser threshold. The resonator length $L$ defines the frequency spacing of the
modes, so mode $n$ has the frequiency 
\begin{equation}
 \omega_n = n \; \pi c / L \quad.
\end{equation}
The output field of a multi-mode laser is the sum of its
modes:
\begin{equation}
  E(t) = \sum_n E_n(t) = \sum_n \; \hat{E}_n \; e^{i \phi_n} e^{i (c/L) \pi n \; t} \quad .
\end{equation}
The temporal evolution of the initial field is thus  the Fourier transform of the  spectral amplitudes $\hat{E}_n$ multiplied by a phase factor. The pulse period  $T = 2L /c$  corresponds to the round-trip  time in the resonator.
If no further precautions are taken, the
phase $\phi_n$ of each mode take a random  and fluctuating value. This
results\footcite{DielsRudolph1996} in a  pattern of incoherent light, repeating with the period $T$.
However, if the individual modes are coupled, i.e., if they 
have a fixed phase relationship to each other,
\begin{equation}
   \delta \phi = \phi_{n+1} - \phi_n = \text{const.} \quad,
   \label{gl_theo_nlo_ml_phiconst}
\end{equation}
then the temporal evolution of the electric field corresponds to the
Fourier transforms the amplitude distribution. The constant
$\delta \phi$ leads to a shift in  time. A typical resonator length is $L = 2$~m. At
a center wavelength of $\lambda_0 = 800$~nm and a
spectral width of the amplification range (limited through the
wavelength selective Lyot filters) of $\Delta \lambda =
4$~nm  about
\begin{equation}
 N \approx \frac{\delta \lambda}{ \lambda^2} \; \frac{L}{\pi} =
 3980
\end{equation}
modes are above the gain threshold. It is thus justified to assume  a
continuous amplitude distribution $\hat{E}(\omega)$.

As an example, we discuss a Gaussian distribution, others are found in the literature. \footcite{DielsRudolph1996,Rulliere2005}.
We write the spectral amplitude distribution $\hat{E}(\omega)$ as
\begin{equation}
  \hat{E}(\omega) = \hat{E}_0 \; \sqrt{\pi} \; \tau_G \; e^{-
  \frac{1}{4} \; (\omega - \omega_0)^2 \; \tau_G^2} \quad .
\end{equation}
The temporal evolution of the electric field follows from this by
Fourier transformation
\begin{equation}
  \hat {E}(t) = \hat{E}_0 \; e^{- ( t / \tau_G ) ^2}
  \qquad\text{and}\qquad E(t) = \hat{E}(t)\; e^{i \omega_0 \, t} \quad .
\end{equation}
The pulse width $\tau_p$ is the full width at half maximum (FWHM) of the electric field envelop
\begin{equation}
  \tau_p = \sqrt{2 \; \ln 2} \; \tau_G \quad.
\end{equation}
The spectral width of the laser is expressed as FWHM $\Delta
\nu$ of the laser spectrum $S(\omega) =
|\hat{E}(\omega)|^2$ and
\begin{equation}
  \delta \nu = \frac{\delta \omega}{2 \pi} = \frac{\sqrt{8 \; \ln
  2}}{2 \pi \; \tau_G} \quad.
\end{equation}
The Fourier transform relates these two quantities and leads to a time-bandwidth-product
\begin{equation}
  \delta \nu \; \tau_p = \frac{2 \; \ln 2}{\pi} \approx 0.441 \quad .
\end{equation}
Laser pulses whose time-bandwidth product reach this value
are called Fourier limited. If the available
 spectral bandwidth $\Delta \nu$ is not used
to get the shortest possible pulses, i.e. the smallest $\tau_p$,
then the measured time-bandwidth product exceeds the
value of 0.441. Provided that pulse shape is Gaussian, the time-bandwidth product tells whether the 
laser system is adjusted optimally .

\section{Dispersion}

How doe sit come that the time-bandwidth product is increased? 
Let us investuigate the  influence of a time-dependent phase $\phi(t)$
of the electric field. For simplification we 
assume that $\phi(t)$ is a polynomial in $t$. This means that
\begin{equation}
  E(t) = \hat{E}(t) \; e^{i ( \omega_0 \; t + \phi(t))}
  = \hat{E}(t) \; e^{i ( \omega_0 \; + d\phi(t)/dt) \; t} \; e^{i
  \; \phi_0}  \quad . \label{gl_theo_nlo_ml_field_with_phase}
\end{equation}
As long as $\phi(t)$ is only linearly dependent on the time $t$, the effect is
 only a shift the central frequency $\omega_0$ to
$\omega_0' = \omega_0 + d\phi(t)/dt$.  In frequency space -- through
Fourier transform of the above equation --  this means that the
phase $\phi$ depends only linearly on the frequency $\omega$. This
is what we required already above by  
$\delta \phi =$~const. . For a square
dependence of the phase on the time a temporal variation occurs
in the central frequency $\omega_0$ of the pulse. One  speaks
of \emph{chirp}, because in the acoustic analogy of a
rising or falling tone sequence. Square
(and higher) dependence of the phase on time (and thus on
of frequency) is determined by the group velocity dispersion
(GVD) in optical elements,
e.g. glasses, but also mirrors. This is associated with the
first and second derivative of the refractive index with 
wavelength. Assuming $\phi = - a \;
t^2 / \tau_G^2$ we then get
\begin{equation}
  \hat{E}(t) = \hat{E}_0 e^{- (1 + i \; a) \; ( t / \tau_G ) ^2}
\end{equation}
and for the time-bandwidth product
\begin{equation}
  \delta \nu \; \tau_p = \frac{2 \; \ln 2}{\pi} \; \sqrt{1 + a^2} \approx
  0.441 \; \sqrt{1 + a^2} \quad ,
\end{equation}
as the spectral bandwidth $\Delta \nu$ is just increased by a factor of
$\sqrt{1 + a^2}$. 

\textbf{Higher orders ? }

\begin{figure}
\center
\includegraphics[width=\textwidth]{\currfiledir theo_nlo_ml_index_of_refraction.pdf}
\caption{The refractive index of the glass BK7 and its derivatives
(data from ref. \cite{DielsRudolph1996}). The
positive second derivative of the refractive index dominates the
group velocity dispersion (GVD). }
\label{fig_pulses_ior_gvd}
\end{figure}



Group velocity dispersion can not be avoided  in a
optical resonator . How transmitting and
also reflective elements change the phase of the
electromagnetic wave  depends strongly on the wavelength. In order to still achieve an optimal, i.e. Fourier limited pulse
the entire GVD must be as wavelength-independent as possible. The influence of the individual elements should therefore 
compensate  each other. Figure
\ref{fig_pulses_ior_gvd} shows the refractive index
of BK7 glass and its derivatives. The positive second derivative
is dominating  the GVD. To compensate for this, a section with
negative GVD is needed in the resonator. This can be  achived by two prisms as shown in figure
\ref{fig_pulses_prism}. Two equal
prisms are placed in front of a mirror such 
that all spectral components hit the mirror perpendicularly
and  the resonator is closed for all spectral components.  

 Detailed calculations \footcite{DielsRudolph1996},
show that the pure geometric path length difference causes 
negative GVD. The path inside the prisms contributes  positive GVD.  By varying the position $z$ one can obtain almost arbitrary values of net GVD of this prism system.


\begin{marginfigure}
\center
\includegraphics{\currfiledir theo_nlo_ml_prismen.pdf}
\caption{prismatic section
group velocity dispersion can be set.}
\label{fig_pulses_prism}
\end{marginfigure}

\section{Autocorrelation}

A pulse width $\tau_P$ in the range of a few pico- or femto-seconds can not be measured electronically, as the involved frequencies  would be too high. The idea is to measure the pulse length optically and using the same laser pulse to sample itself. The beam is divided into two parts, one of which is delayed with respect to the other before they are overlaid again.
 Times in the 
range of femtoseconds correspond to lengths in the range of
micrometers, which can be easily controlled. A detector that responds quadratically in intensity (not field!) is able to measure  overlap of two halves $a$ and $b$ :
\begin{equation}
 \ \text{after each other} \qquad a^2 + b^2 \neq (a+b)^2 \qquad
 \text{simultaneously.}
\end{equation}
Often  frequency doubling is used for this purpose by focusing the
recombined beam onto a frequency doubling crystal.
The detector is sensitive only for light at half the wavelength. With variable delay $\tau$ the time-integrated signal is sufficient and  the temporal resolution
of the detector is  irrelevant. One thus measures
\begin{equation}
  G(\tau) = \left< I(t) \times I(t-\tau) \right> \quad.
\end{equation}
The function $G(\tau)$ is called as intensity autocorrelation function. It is always symmetrical ($G(\tau) = G(-\tau)$) and
is relatively insensitive to the actual pulse shape.  One has to assume a pulse as Gaussian
 or
$\text{sec}^2$-shaped and can then calculate\footcite{DielsRudolph1996}: the pulse length $\tau_{P} $ from the 
FWHM of the autocorrelation
$\tau_{AC}$ 
\begin{align}
  \tau_{P} &= \tau_{AC} / \sqrt{2} &\quad& \text{for Gauss pulses,} \\
           &= \tau_{AC} / 1,543 && \text{for $\text{sec}^2$-shaped pulses.}
\end{align}

		
\printbibliography[segment=\therefsegment,heading=subbibliography]

