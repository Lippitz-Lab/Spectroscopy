
\section{parametric amplification}

By parametric amplification we mean a process in which
a vibration is not driven from the outside, but by
periodic variation of a parameter is increased
\cite{Milonni_lasers}:
\begin{alignat}{3}
  \text{getrieben:} &\qquad & \ddot{x} &\;+\; & \omega_0^2 \; x &\;=\;  F(t)  \\%
  \text{parametrisch:} && \ddot{x} &\;+\;& \omega^2(t) \; x &\;=\;   0
\end{alignat}
A simple example of parametric gain is a
Pendulum whose pendulum length periodically moves at the frequency $\omega'$
is varied. In this case the already
existing oscillation, if $\omega' \approx 2 \omega_0$. If
the pendulum is completely at rest at the beginning, so $x \equiv 0$,
the variation of the pendulum length has no influence, it can
Just amplify the vibration. Opposite, a driven pendulum can lead to
Begin to be quite calm.

\begin{figure}[b]
\center
\includegraphics[scale=1.95]{\currfiledir theo_nlo_param_vers2.pdf}
\caption{Parametric Gain
\cite{saleh_fundamentals_of_photonics}. A: in the presence of a
pump wave, the signal wave is amplified and an Idler wave
is generated. B: The energy of a pump photon is converted to a
signal and one eagle photo. C: Signal and
Idler intensity show the same course with increasing
crystal thickness $z$, where no idler wave is present at the beginning
is.} \label{abb_theo_nlo_param_verst}
\end{figure}

A strong laser field of the frequency $\omega_L$ provides in a
nonlinear crystal for the variation of a parameter which
Dielectric constants $\epsilon$, because \cite{kroll62}
\begin{equation}
  P = \epsilon_0 \; \chi(E) \; E = \epsilon_0 \;(\epsilon(E) -1)\; E = \epsilon_0
  \left( \chi^{(1)} E + \chi^{(2)} E^2 + \cdots \right) \quad .
\end{equation}
This gives the following process its name. In the presence of a
strong pump wave $E_3$ the so-called signal wave $E_1$
and creates a so called idler wave $E_2$, because this
is necessary for energy and impulse conservation:
\begin{equation}
   \omega_1 + \omega_2 = \omega_3 \qquad \text{and} \qquad \delta
   k = k_1 + k_2 - k_3 \quad .
\end{equation}
With optimum phase matching ($\Delta k = 0$) and under the
Assumption that the pump intensity over the length of the crystal
is constant ($\hat{E}_3(z) = \text{const.}$), followed by
Integration
\begin{eqnarray}
  \left| \hat{E}_1(z) \right|^2 & = & \left| \hat{E}_1(0)
  \right|^2 \; \cosh^2 K z \quad ,\\\
  %
    \left| \hat{E}_2(z) \right|^2 & = & \frac{\omega_1}{\omega_2} \; \left| \hat{E}_1(0)
  \right|^2 \; \sinh^2 K z \quad,
\end{eqnarray}
where $K \propto \chi^{(2)} \; \left| \hat{E}_3(0) \right|$.
This situation is shown in figure \ref{abb_theo_nlo_param_verst}
summarized. The intensity of the Signal and Idler wave increases
so almost square with the length $z$.


\begin{figure}[b]
\center
\includegraphics[scale=0.99]{\currfiledir  theo_nlo_opo.pdf}
\caption{Phase adjustment and tuning of a
optical parametric oscillator (OPO) with an
Potassium titanyl arsenate crystal (KTA, KTiOAsO$_4$). A: Course of the
Refractive indices for ordinary and extraordinary waves at
non-critical phase matching ($\theta = 90$ deg, $\phi = 0$deg). B:
The signal wavelength as a function of the pump wavelength. This
Connection is given by the refractive indices. C: Only for a
pair of signal and eagle energies is simultaneously energy and
Maintaining impulse. } \label{abb_theo_nlo_opo}
\end{figure}

In an optical parametric oscillator (OPO) this
parametric amplification is used as a laser medium, which is
a pump field $E_3$ is supplied with energy. For a given
frequency $\omega_3$ of the pump laser decides the
Phase matching, which of the photon pairs of frequency $\{
\omega_1, \omega_2 \}$ which fulfil the energy conservation
will be. The first signal-photon, which is then amplified
through so-called parametric fluorescence analogous to spontaneous
emission in a conventional laser \cite{Milonni_lasers}.
Figure \ref{abb_theo_nlo_opo} shows this for a
Potassium titanyl arsenate crystal (KTA, KTiOAsO$_4$), as also found in the
is used here (see section
\ref{kap_aufb_laser_opo}) Since KTA is biaxial, so two
has excellent optical axes, can be
orientation of the crystal the refractive index of the ordinary and
of the extraordinary beam is influenced in a wide range
will be. On the same grounds as
Frequency doubling is also the KTA crystal non-critical
phase matched ($\theta = 90$deg, $\phi = 0$deg). The refractive indices
are only slightly dependent on the temperature at room temperature, so
that the crystal does not need to be thermostatted. Part C of the
Figure sketches how through different
Polarization directions of signal and idler wave Energy and
pulse conservation for a frequency pair are fulfilled simultaneously.
This results in the dependence of the signal wavelength of
the pump wavelength as shown in part B. In the Annex
\ref{kap_anhang_kta} is summarized as follows
Sellmeier coefficients for describing the refractive index these
curve.

\section{Kerr effect}

As a third-order nonlinear effect, the
the Kerr effect played a role. It is supposed to
are briefly presented here, even if it is not intended for the production of
of new frequency components. In an inversion-symmetrical
medium disappears $\chi^{(2)}$, so that the first two links
from Gl. \ref{gl_theo_nlo_polarisation_simple} are
\begin{equation}
  P(E) = \epsilon_0 \; \left( \chi^{(1)} E + \chi^{(3)} E^3 \right)
  = \epsilon_0 \; \left( \chi^{(1)} E + \chi^{(3)} \left|E \right|^2 \; E
  \right) = \epsilon_{\text{total}} \; E \quad .
\end{equation}
This allows an intensity-dependent refractive index
$n_{\text{total}}$ are formed \cite{yariv_QE}:
\begin{equation}
 n_{\text{total}} = \sqrt{\epsilon_{\text{total}} } \approx n +
    n^{(2)} \; \left|E \right|^2 \quad.
\end{equation}
If two of the four are connected to this nonlinear effect of a third
order DC voltage fields are used
this Kerr effect is used to modulate the
refractive index is used in variable polarizers, for example. The
The modulation depth is then square to the applied
Voltage. However, if the laser beam is intense enough
he can modulate the refractive index himself. High
Intensities in the middle of the beam profile cause a high
refractive index, which is equivalent to a lens of positive
is focal length.



