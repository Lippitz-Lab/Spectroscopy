\documentclass[a4paper]{article}

\usepackage[latin1]{inputenc}
\usepackage{amsmath}
\usepackage[dvips]{graphicx}

\setlength{\parindent}{0pt} \setlength{\parskip}{1em}
\title{Single molecule spectroscopy of vibronic levels?}

\begin{document}

\maketitle

\section{Time-dependent perturbation theory}

Quantummechanical system:

\begin{tabular}{l|lll|ll}
state & atom & elec. & vibr. & light $\omega_1$ & light
$\omega_2$ \\ \hline%
initial $|i>$ & ground  $|g>$ & 0  & 0 & a  & b  \\%
intermediate $|x>$ & excited  $|e>$& 1 & 0  & a-1 & b \\%
intermediate' $|x'>$ & excited '  $|e'>$& 1 & 1& a-1  & b \\ %
final $|f>$ & vibronic  $|v>$ & 0 &1& a-1 &b+1
\end{tabular}


wavefunction $\psi$:
\begin{equation}
\psi = \sum_{\mu \in \{i, x, x', f\}} c_{\mu} \psi_{\mu}
\end{equation}

initially system in ground state: $c_i(0) = 1$, all other
$c_{\mu}(0) = 0$.


wanted: population of vibronic level v  $|c_f(t)|^2$


Hamiltonian: $H = H^0 + H^S$

time evolution of matrix elements
\begin{equation}
H^S_{\mu \nu} (t) = H^S_{\mu \nu} (0) \; e^{i \, \omega_{\mu \nu}
t}
\end{equation}
with $\hbar  \omega_{\mu \nu} = E_{\mu} - E_{\nu}$

with time-dependent perturbation theory, including only processes
involving two photons (see, for example, Haken / Wolf:
Molek�lphysik und Quantenchemie, p.353)
\begin{equation}
c_f(t) = - \frac{1}{\hbar^2} \sum_{\mu} H^S_{f \mu} (0) H^S_{\mu
i} (0)\int_0^t e^{i \,\omega_{f \mu} t_1} dt_1 \int_0^{t_1} e^{i
\,\omega_{\mu i} t_2} dt_2
\end{equation}


\textbf{evaluation of time integrals}

inner integral
\begin{equation}
\int_0^{t_1} e^{i \,\omega_{\mu i} t_2} dt_2 = \left[ \frac{1}{i
\,\omega_{\mu i}} e^{i \,\omega_{\mu i} t_2} \right]_0^{t_1} =
\frac{1}{i \,\omega_{\mu i}} \left( e^{i \,\omega_{\mu i} t_1} -1
\right)
\end{equation}

outer integral
\begin{equation}
\int_0^t e^{i \,\omega_{f \mu} t_1} \frac{1}{i \,\omega_{\mu i}}
\left( e^{i \,\omega_{\mu i} t_1} -1 \right) dt_1 = %
\int_0^t  \frac{e^{i \,(\omega_{f \mu}+\omega_{ \mu i}) t_1}}{i
\,\omega_{\mu i}} %
-   \frac{e^{i \,\omega_{ f \mu } t_1}}{i \,\omega_{\mu i}} dt_1
\end{equation}

\begin{equation}
= \left[  \frac{e^{i \,\omega_{fi} t_1}}{(-1)
\,\omega_{\mu i} \, \omega_{fi}} %
+   \frac{e^{i \,\omega_{ f \mu } t_1}}{\omega_{f \mu}\,
\omega_{\mu i}} \right]_0^t %
= -  \frac{e^{i \,\omega_{fi} t} -1}{
\omega_{\mu i} \, \omega_{fi} } %
+   \frac{e^{i \,\omega_{ f \mu } t} -1}{\omega_{f \mu}\,
\omega_{\mu i}}  %
\end{equation}
where $\omega_{f \mu}+\omega_{ \mu i}= \omega_{f i}$ is approx.
zero (guarantees energy conservation), but $\omega_{f \mu}$ is
large: the sinc$^2$ oscillates fast $\rightarrow$ the second term
is neglected (rotating wave approximation)

Now we have
\begin{equation}
c_f(t) = (-1) \cdot (-1) \frac{1}{\hbar} \frac{e^{i \,\omega_{fi}
t} -1}{ \omega_{fi} } \sum_{\mu} \frac{H^S_{f \mu} (0) H^S_{\mu i}
(0)}{E_{\mu} - E_i}
\end{equation}


\textbf{Matrix elements}


$E_n$ is light field of wave $n$, $F_a^b$ is Frank-Condon-Integral
from vibronic level a in the electronic ground state to b in the
electronic excited state. $\rightarrow$ $F_0^0 = F_1^1 \approx 1$,
$F_0^1 = - F_1^0 < 0$. $\mu_{eg}$ is the dipol moment of the
transition g $\rightarrow$ e.

\begin{tabular}{llll}
  & elec. & vibr. & total\\%
$H^S_{x i}$  & 0 $\rightarrow$ 1 &  0 $\rightarrow$ 0 &  $\mu_{eg} \, E_1 \, F_0^0$ \\%
$H^S_{x' i}$  & 0 $\rightarrow$ 1 &  0 $\rightarrow$ 1 &  $\mu_{eg} \, E_1 \, F_0^1$ \\%
$H^S_{f x}$  & 1 $\rightarrow$ 0 &  0 $\rightarrow$ 1 &  $\mu_{eg}^{\star} \, E_2 \, F_1^0$ \\%
$H^S_{f x'}$  & 1 $\rightarrow$ 0 &  1 $\rightarrow$ 1 &  $\mu_{eg}^{\star} \, E_2 \, F_1^1$ \\%
\end{tabular}

\textbf{Energy differences}

\begin{eqnarray}
E_x - E_i  & =  &\left(E_e + (a-1) \hbar \omega_1 + b \hbar
\omega_2) \right) - \left(E_g + a \hbar \omega_1 + b \hbar
\omega_2) \right) \\
&= & E_e - E_g - \hbar \omega_1 = \hbar (\omega_{eg} - \omega_1)
\end{eqnarray}


\begin{eqnarray}
E_{x'} - E_i  & =  &\left(E_{e'} + (a-1) \hbar \omega_1 + b \hbar
\omega_2) \right) - \left(E_g + a \hbar \omega_1 + b \hbar
\omega_2) \right) \\
&= & E_{e'} - E_g - \hbar \omega_1 = \hbar ( \omega_{eg} -
\omega_1 + \Omega_v)
\end{eqnarray}

$\mathbf{\Rightarrow}$
%
\begin{eqnarray}
\sum_{\mu} \frac{H^S_{f \mu} (0) H^S_{\mu i} (0)}{E_{\mu} - E_i} &
= &
 \frac{H^S_{f x}  H^S_{x i} }{E_{x} - E_i} +
 \frac{H^S_{f x'} H^S_{x' i} }{E_{x'} - E_i} \\%
& = & \frac{ |\mu_{eg}|^2 \, E_1 \, E_2 \, F_0^0 \, F_1^0}{\hbar
(\omega_{eg} - \omega_1)} +%
 \frac{ |\mu_{eg}|^2 \, E_1 \, E_2 \,
F_1^1 \, F_0^1}{\hbar (\omega_{eg} - \omega_1 + \Omega_v)} \\%
& = & \frac{ |\mu_{eg}|^2 \, E_1 \, E_2  \, F_0^1}{\hbar} \left(
\frac{1}{ \omega_{eg} - \omega_1 + \Omega_v} -
 \frac{ 1}{\omega_{eg} - \omega_1 } \right) \\%
& = & \frac{ |\mu_{eg}|^2 \, E_1 \, E_2  \, F_0^1}{\hbar}%
 \frac{- \Omega_v}{
(\omega_{eg} - \omega_1) (\omega_{eg} - \omega_1 + \Omega_v)} \\%
& = & \frac{ |\mu_{eg}|^2 \, E_1 \, E_2  \, F_0^1}{\hbar}%
 \frac{-\Omega_v}{ \delta_1 (\delta_1 + \Omega_v)}
\end{eqnarray}
with $\delta_1 = \omega_{eg} - \omega_1$ and $\delta_1 \gg
\Omega_v$, so that $\delta_1 (\delta_1 + \Omega_v) \approx
(\delta_1)^2$.


Now we have
\begin{equation}
c_f(t) =
\frac{1}{\hbar} \frac{e^{i \,\omega_{fi} t} -1}{ \omega_{fi} } \; \cdot\; \frac{ |\mu_{eg}|^2 \, E_1 \, E_2  \, F_0^1}{\hbar}%
 \frac{-\Omega_v}{(\delta_1)^2 } %
 = \frac{1}{\hbar} \frac{e^{i \,\omega_{fi} t} -1}{ \omega_{fi} } \;
 \cdot\;X_{fi}
\end{equation}

\begin{equation}
|c_f(t)|^2 = \frac{|X_{fi}|^2}{\hbar^2} \cdot t^2 \cdot
\frac{\sin^2(\omega_{fi} \,t / 2)}{ \left[\omega_{fi} \,t / 2
\right]^2}
\end{equation}


\begin{eqnarray}
\hbar \omega_{fi} & = & E_f - E_i \\%
&= &(E_v + (a-1) \; \hbar \omega_1 + (b+1) \; \hbar \omega_2) -
(E_g + a \; \hbar \omega_1 + b \;
\hbar \omega_2) \\%
 & = & E_v - E_g - \hbar \omega_1 + \hbar \omega_2 = \hbar (
 \Omega_v - \omega_{12} )
\end{eqnarray}

If the line widths would be infinitesimal narrow and the laser
tuned on resonance ($\Omega_v = \omega_{12}$), the population of
$|v>$ would increase quadratically in time. Now we assume a
Lorenzian line shape with a line width of $\Gamma_v$ (better:
laser line width convoluted with transition line width is
$\Gamma_v$).

We assume very narrow laser lines tuned to the center of $|v>$,
Line shape function (or Density of states, as in Meystre / Sargent
III, p.62)
\begin{equation}
D(\omega) = \frac{\Gamma_v / \pi}{(\omega_{12} - \omega)^2 +
\Gamma_v^2}
\end{equation}

population of $|v>$ with line width $\Gamma_v$:
\begin{equation}
P_v = \int D(\omega) \frac{|X_{fi}(\omega)|^2}{\hbar^2} \; t^2 \;
\text{sinc}^2\left[ (\omega - \omega_{12}) \,t / 2 \right] \;
d\omega
\end{equation}

Case A: very short times, so that $(\omega - \omega_{12}) t \ll
1$, which means $\Gamma_v t \ll 1$. Then $\text{sinc}^2(\Gamma t /
2) \approx 1$, so that $P_T \propto t^2$.

Case B: longer times. $\text{sinc}^2[(\omega - \omega_{12}) t / 2]
> 0$ only if $\omega = \omega_{12}$.
\begin{eqnarray}
P_v  &= & \int D(\omega) \frac{|X_{fi}(\omega)|^2}{\hbar^2} \; t^2
\; \text{sinc}^2\left[ (\omega - \omega_{12}) \,t / 2 \right]
\; d\omega \\%
& = &  D(\omega_{12}) \frac{|X_{fi}(\omega_{12})|^2}{ \hbar^2} \;
t^2 \; \int \text{sinc}^2\left[x \right]
\; \frac{2}{t} dx\\%
& = &  \frac{1}{\pi \Gamma_v} \frac{|X_{fi}(\omega_{12})|^2}{
\hbar^2} \; t^2  \; \frac{2}{t} \; \pi \\%
 & = &  \frac{2 t}{ \Gamma_v}
 \left|
 \frac{ |\mu_{eg}|^2 \, E_1 \, E_2  \, F_0^1}{\hbar^2}%
 \frac{-\Omega_v}{(\delta_1)^2 } \right|^2%
\end{eqnarray}


\section{Michels calculations (31. Jan. 01)}

\textbf{populations and coherences with Bloch equations}

Bloch equations for two-state system in rotating frame (see,for
example, Loudon, p. 78)
\begin{eqnarray}
\dot{\rho}_{22} & = & - \frac{1}{2} i V^{\star} \rho_{12}
 + \frac{1}{2} i V \rho_{21} - 2 \gamma \rho_{22} \\%
 \dot{\rho}_{21} & = & - \frac{1}{2} i V^{\star} ( \rho_{11} -
 \rho_{22}) -  \gamma' \rho_{21}
\end{eqnarray}
where $2 \gamma = 1 /  T_1$, $\gamma' = 1 / T_2$, $V = \mu E /
\hbar$, $\rho_{12} = \rho_{21}^{\star}$, $\rho_{11} + \rho_{22} =
1$.

Assumptions: $\rho_{11} \approx 1$, $\rho_{22} \ll 1$. coherences
constant after $t > T_2$: $\dot{\rho}_{21} = 0$
\begin{equation}
\rightarrow \quad \rho_{21} = - i \; \frac{V^{\star}}{2 \gamma'}
\qquad \text{and} \qquad \rho_{12} =  i \; \frac{V}{2 \gamma'}
\end{equation}


\begin{eqnarray}
\rightarrow \quad \dot{\rho}_{22} & = & - \frac{1}{2} i V^{\star}
i \; \frac{V}{2 \gamma'}
 + \frac{1}{2} i V - i \; \frac{V^{\star}}{2 \gamma'} - 2 \gamma  \; \cdot \; 0
 \\%
 & = & \frac{1}{4} \frac{|V|^2}{\gamma'} +  \frac{1}{4}
 \frac{|V|^2}{\gamma'} =  \frac{1}{2} \frac{|V|^2}{\gamma'} =
 |\rho_{21}|^2 \; 2\; \gamma'
 \end{eqnarray}


Integration yields population of $|v>$:
\begin{equation}
P_{22} = t \; 2 \; \gamma' \;|\rho_{21}|^2
\end{equation}


The coherence $\rho_{21}$ from Michels calculations:
\begin{equation}
|\rho_{21}| = \frac{ \mu_{eg}^2 E_1 E_2 F_0^1}{\hbar^2 \Gamma_v}
\; \frac{\Omega_v}{(\delta_1)^2}
\end{equation}


At this point we use
\begin{equation}
 \Gamma_v = \frac{1}{T_2} = \frac{1}{2 T_1} +
 \frac{1}{T_2^{\star}} \approx  \frac{1}{T_2^{\star}} = \gamma'
\end{equation}
because $T_2 \approx T_2^{\star} \ll T_1$.

All together
\begin{equation}
P_{22} = 2 \; t \; \Gamma_v \; \left[ \frac{ \mu_{eg}^2 E_1 E_2
F_0^1}{\hbar^2 \Gamma_v} \; \frac{\Omega_v}{(\delta_1)^2}
\right]^2
\end{equation}


\section{Numerical values}

\textbf{Transition dipole moment}

from radiative rate:
\begin{equation}
\gamma_{\text{radiative}} = \frac{1}{\tau_{fl}} = \frac{4}{3}
|\mu|^2 \frac{1}{4 \pi \epsilon_0 \hbar}\left( \frac{\omega_0}{c}
\right)^3
\end{equation}
Difference to classical total emitted power from dipole: factor 2
$\rightarrow$ QM Weisskopf -- Wigner spontaneous emission decay
rate (see Meystre / Sargent III, p.300)
\begin{equation}
\rightarrow \quad |\mu|^2 = 3 \pi \epsilon_0 \hbar \left(
\frac{\lambda}{2 \pi} \right)^3 \; \frac{1}{\tau_{\text{fl}}}
\end{equation}


\textbf{Electric fields}

Focused pulsed laser
\begin{equation}
I = \frac{P}{f \; \tau \; r^2 \; \pi / 2} \qquad \text{and} \qquad
E = \sqrt{ \frac{2 \; I}{\epsilon_0 \; c}}
\end{equation}
where $P$ laser power, $f$ rep. rate, $\tau$ pulse width, $r$ spot
radius ($1/e$ of the intensity) (Factor of 2 because we the need
peak value of the field, but have the average value of the
intensity)


\begin{tabular}{lr}
\begin{minipage}{5cm}
\begin{tabular}{ll}
parameter & \\ \hline
$\lambda$ & 570 nm \\
$\tau_{fl}$ & 4 ns \\
$P_1 = P_2$ & 100 mW \\
f & 76 MHz \\
$\tau$ & 1 ps \\
r & 3 $\mu$m \\
$F_0^1$ & 0.5 \\
$\Omega_v$ & 1500 cm$^{-1}$ \\
$\Gamma_v$ & 10 cm$^{-1}$ \\
$\delta_1$ & 8000 cm$^{-1}$
\end{tabular}
\end{minipage}&%
\begin{minipage}{5cm}
\begin{tabular}{ll}
results & \\ \hline
$\mu_{eg}$ & 4 10$^{29}$ Cm \\
$E_1 = E_2$ & 2.6 10$^8$ V/m \\
$\rho_{21}$& 0.29 \\
$P_v$ & 0.31
\end{tabular}
\end{minipage}
\end{tabular}

One third of  the molecules are in the excited vibronic state.
This value seems to be too high for the validity of perturbation
theory or for the approximation $\rho_{22} \approx 0$ in the Bloch
equation. In reality, the depopulation of the ground state should
reduce this value. Nevertheless it should be in the order of some
10~\%.


\textbf{Estimation of $I_3$}


Rabi oscillations, maximum population, but not oscillating (see
Loudon, Fig. 2.7, p.67)
\begin{equation}
  \Omega_3 / \Gamma_3 \approx 1
\end{equation}

$\Omega_3 = \Gamma_3 \approx 500 \text{cm}^1$ (20 nm at 600 nm)

\begin{equation}
\rightarrow \quad \Omega_3 \approx \Omega_1 = \frac{\mu_{eg} \;
E_1}{\hbar} \quad \rightarrow \quad E_3 \approx E_1
\end{equation}

Laser beam 3 (approx. 720 nm) is focused by high NA objective to
get high spacial resolution. A spot size of 300 nm should be
possible = factor 10 better than beam 1 and 2 $\rightarrow$ factor
100 less laser power = about 1 mW.


\end{document}
