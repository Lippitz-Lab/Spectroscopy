%\renewcommand{\lastmod}{June 15, 2020}
\chapter{Two-Photon Absorption}


\section{Tasks}

\begin{itemize}
\item Derive eq. \ref{eq:tpe_vib_ex_final} which is the same as the result of section 2.3.(i) in \cite{Winterhalder11} and plug in some typical values. The paper investigates the possibility of single molecule vibrational spectroscopy by stimulated Raman scattering followed by a 'normal' electronic excitation. The interesting part is the first half of the process, a two-photon excitation of a vibrational state. The technique to 'read' the formula from the interaction diagram will be discussed in the chapter on 2d spectroscopy.
\end{itemize}



\section{History of Two-Photon Absorption}

The possibility of simultaneous absorption of two photons in
a quantum process was  described already  in 1931  by Maria Göppert-Mayer in
of her dissertation\footcite[About elementary acts with two quantum jumps]{goeppert31}  'Über Elementarakte mit zwei Quantensprüngen'. One year
after the realization of the ruby laser by Maiman
\footcite{maiman60} succeeded in 1961 the first experimental proof of
two-photon excitation of CaF$_2$:Eu$^{2+}$ by Kaiser and
Garrett \footcite{Kaiser61}. Two years later, fluorescence of 
organic crystals (pyrene, anthracene) was observed after 
two-photon excitation  \footcite{peticolas63}.  Important
fields of application of two-photon absorption are
Doppler-free spectroscopy of atoms and molecules in the
molecular beam (introduction in 
\cite{Demtroeder_laser}) and the optical
spectroscopy of transitions which are forbidden for a single-photon process \footcite{birge86}. The development of 
two-photon microscopy with the aim of detecting individual
molecules began with the work of  \cite{denk90}. Nowadays, two-photon absorption is widely used in live science for imaging in tissues.





\section{Second-order perturbation theory}

\begin{marginfigure}
%\includegraphics[width=\textwidth]{\currfiledir theo_nlo_tpe_qm_system.pdf}
\inputtikz{\currfiledir tpe_states}

\caption{
We distinguish between the states of the absorbing system
($\ket{g}$, $\ket{e'}$, $\ket{e}$) and the states of the whole system, which
also includes the light field with initially $n$ photons: $\ket{i}$, $\ket{x}$, $\ket{f}$. }
\label{fig:tpe_qm_system}
\end{marginfigure}

In this section, we derive an expression for
the absorption rate of a two-photon transition using 
time-dependent perturbation theory
\footcite{Haken_wolf_II,mystre_quantum_optics}.
Figure \ref{fig:tpe_qm_system} outlines the involved
states. We only  consider the case
that both absorbed photons come from the same monochromatic
wave of the frequency $\omega$. The intermediate state 
$\ket{e'}$
does not necessarily have to be energetically below the final state $\ket{e}$.
The wave function $\psi$ of the entire system is
\begin{equation}
\psi = \sum_{\mu \in \{i, x, f\}} c_{\mu} \psi_{\mu} \quad.
\end{equation}
At the beginning the system is in the ground state, so $c_i(0)
= 1$ and all others $c_{\mu}(0) = 0$. We are looking for the population
of the final state $\ket{f}$ depending on the time $t$, i.e.,
$|c_f(t)|^2$. The Hamilton operator $H$ is split in a fixed and a perturbing part as
 $H = H^0 + H^S$ . 
The matrix elements of the perturbing part evolve in time as
\begin{equation}
H^S_{\mu \nu} (t) = H^S_{\mu \nu} (0) \; e^{i \, \omega_{\mu \nu}
t}
\end{equation}
where $\hbar \omega_{\mu \nu} = E_{\mu} - E_{\nu}$.
We now only consider processes that require the absorption of two photons at times $t_1$ and $t_2$. We get\footcite{Haken_wolf_II}
\begin{equation}
c_f(t) = - \frac{1}{\hbar^2} \sum_{\mu} H^S_{f \mu} (0) H^S_{\mu
i} (0)\int_0^t e^{i \,\omega_{f \mu} t_1} dt_1 \int_0^{t_1} e^{i
\,\omega_{\mu i} t_2} dt_2 \quad .
\end{equation}
The inner integral gives
\begin{equation}
\int_0^{t_1} e^{i \,\omega_{\mu i} t_2} dt_2 =
% \left[ \frac{1}{i \,\omega_{\mu i}} e^{i \,\omega_{\mu i} t_2} \right]_0^{t_1} =
\frac{1}{i \,\omega_{\mu i}} \left( e^{i \,\omega_{\mu i} t_1} -1
\right)
\end{equation}
and the outer integral
\begin{equation}
\int_0^t e^{i \,\omega_{f \mu} t_1} \frac{1}{i \,\omega_{\mu i}}
\left( e^{i \,\omega_{\mu i} t_1} -1 \right) dt_1 = %
\int_0^t  \frac{e^{i \,(\omega_{f \mu}+\omega_{ \mu i}) t_1}}{i
\,\omega_{\mu i}} %
-   \frac{e^{i \,\omega_{ f \mu } t_1}}{i \,\omega_{\mu i}} dt_1
\end{equation}

\begin{equation}
= \left[  \frac{e^{i \,\omega_{fi} t_1}}{(-1)
\,\omega_{\mu i} \, \omega_{fi}} %
+   \frac{e^{i \,\omega_{ f \mu } t_1}}{\omega_{f \mu}\,
\omega_{\mu i}} \right]_0^t %
= -  \frac{e^{i \,\omega_{fi} t} -1}{
\omega_{\mu i} \, \omega_{fi} } %
+   \frac{e^{i \,\omega_{ f \mu } t} -1}{\omega_{f \mu}\,
\omega_{\mu i}}  %
\label{eq:tpe_outer_integral}
\end{equation}
While $\omega_{f \mu}+\omega_{ \mu i}= (E_e - E_g)/ \hbar - 2 \omega = \Delta \omega \approx 0$ guarantees  energy conservation,  terms with 
 $\omega_{f \mu} \gg 1$ oscillate so fast that they
 average out and can be neglected. This is again the rotating wave approximation. This   results in
\begin{equation}
c_f(t) =  \frac{1}{\hbar} \frac{e^{i \,\Delta \omega t} -1}{
\Delta \omega } \sum_{\mu} \frac{H^S_{f \mu} (0) H^S_{\mu i}
(0)}{E_{\mu} - E_i} =  \frac{1}{\hbar} \frac{e^{i \,\Delta \omega t}
-1}{ \Delta \omega } \; X_{fi}
\end{equation}
and
\begin{equation}
|c_f(t)|^2 = \frac{|X_{fi}|^2}{\hbar^2} \cdot t^2 \cdot
\text{sinc}^2(\Delta \omega \,t / 2) \quad .
\end{equation}
This is the most interesting point, the definition of an effective transition  matrix element $X_{fi}$
\begin{equation}
X_{fi} = \sum_{\mu} \frac{H^S_{f \mu} (0) H^S_{\mu i}
(0)}{E_{\mu} - E_i}  \quad .
\end{equation}

From now on, we follow the normal route also used when deriving Fermi's Golden Rule. The effect of the second order in the perturbation theory is collapsed into  $X_{fi}$. Again we realize that with
 infinitesimally sharp optical transitions
the transition rate $w_{fi}$ at resonance ($\Delta \omega =
0$) increases linearly with time $t$:
\begin{equation}
 w_{fi} = \frac{d |c_f(t)|^2}{dt} = 2 \frac{|X_{fi}|^2}{\hbar^2} \cdot t \cdot
\text{sinc}^2(\Delta \omega \,t / 2) \quad .
\end{equation}
By assuming a Lorentz-shaped line of width $\Gamma$, whose
form is described by the function $D(\Delta \omega)$,
\begin{equation}
D(\Delta \omega) = \frac{\Gamma / \pi}{(\Delta \omega)^2 +
\Gamma^2}
\end{equation}
the population $P_e$ of $\ket{e}$ can be calculated 
\footcite{mystre_quantum_optics}:
\begin{equation}
P_e = \int D(\Delta \omega) \frac{|X_{fi}|^2}{\hbar^2} \; t^2 \;
\text{sinc}^2\left[ \Delta \omega  \,t / 2 \right] \; d (\Delta
\omega) \quad .
\end{equation}
For times $t \gg 1 / \Gamma$ we find that $\text{sinc}^2\left[ \Delta
\omega \,t / 2 \right]$ is different from zero only for $\Delta \omega = 0$  so that
\begin{equation}
P_e =  \frac{2}{ \Gamma} \frac{|X_{fi}|^2}{ \hbar^2} \; t \qquad
\text{and} \qquad w_{fi} =  \frac{2}{ \Gamma} \frac{|X_{fi}|^2}{
\hbar^2} \quad . \label{eq:tpe_pe_xfi}
\end{equation}

Let us examine the term $|X_{fi}|^2$  more closely. The
matrix elements of the perturbation operator $H^S$ are the product of the
complex dipole matrix element $\mu_{\mu \nu}$ and the amplitude of the electric field
\begin{equation}
  H^S_{\mu \nu} = \hat{E}(\omega) \; \mu_{\mu \nu} \quad .
\end{equation}
The energy difference $E_\mu - E_i$ can also be expressed as $E_u - \hbar
\omega$ ($u \in \{g, e', e \}$) so that
\begin{equation}
|X_{fi}|^2 = \left| \sum_{\mu} \frac{H^S_{f \mu} (0) H^S_{\mu i}
(0)}{E_{\mu} - E_i} \right|^2 = \left| \hat{E}(\omega) \right|^4
\;  \left| \sum_{u} \frac{\mu_{e u} \; \mu_{u g}}{E_{u} - \hbar
\omega} \right|^2 \quad .
\end{equation}
All in all, this results in the following transition rate $w_{fi}$ 
\begin{equation}
w_{fi} =  \frac{2}{\hbar^2 \Gamma} \; \left| \hat{E}(\omega) \right|^4  \; \left|
\sum_{u} \frac{\mu_{e u} \; \mu_{u g}}{E_{u} - \hbar \omega}
\right|^2 = \frac{1}{2} \; \sigma_{TPE} \;
\left(\frac{I(\omega)}{\hbar \; \omega}\right)^2 \quad ,
\label{eq:tpe_transition_rate}
\end{equation}
where $\sigma_{TPE}$ is the two-photon absorption cross section\sidenote{Note that $\sigma_{TPE}$ does \emph{not} have the units of an area! See below.}
and the factor $1/2$ takes into account\footcite{xu97} that two photons
must be absorbed. The
excitation intensity $I$ is  usually not temporally
constant, because it is advantageous to use a pulsed laser. This leads to an average excitation rate $\alpha$
\begin{equation}
  \alpha = \left< w_{fi} \right> =
  \frac{1}{2} \; \sigma_{TPE} \; \frac{ \left<I^2 \right>}{(\hbar \; \omega)^2}
  = \frac{1}{2} \; \sigma_{TPE} \; g^{(2)}(\tau = 0) \;
  \left(\frac{\left< I \right>}{\hbar \; \omega}\right)^2 \quad .
\end{equation}
Here $g^{(2)}(\tau = 0)$ denotes the 
intensity autocorrelation function, evaluated at  time zero. 
Knowing the pulse shape we can calculate 
 $g^{(2)}(\tau
= 0)$ so that
\begin{equation}
  \alpha
  = \frac{1}{2} \; \sigma_{TPE} \; \frac{g_P}{f \; \tau_P} \;
  \left(\frac{\left< I \right>}{\hbar \; \omega}\right)^2 \quad .
\end{equation}
For Gaussian pulses the pulse shape factor $g_P$ has the value
$g_P = 0.66$, for sec$^2$-shaped pulses $g_P = 0.59$ applies
\footcite{xu97}. $f$ denotes the repetition rate of the laser and
$\tau_P$ the temporal FWHM of the pulse. 


\begin{questions}

\item How does it come that $|X_{fi}|^2 $ can be used in the same way independent whether it stems from a first or second order perturbation theory? Why are its units independent of the order?

\item The requirement $\omega_{f \mu} \gg 1$ after eq.\ref{eq:tpe_outer_integral} is formulated a bit sloppy. How can this be improved?

\item What is the difference between $w_{fi} $ and $\alpha$ ? Why does this distinction become necessary?

\end{questions}

\section{Selection rules and cross sections}

For two-photon absorption to take place, the
final state $\ket{e}$ has to be reached  over two successive electrical
dipole transitions from the ground state  $\ket{g}$.  
All intermediate states  $\ket{x}$ of the molecule contribute, weighted  with their energetic
distance to the photon energy.
Since the summation occurs before taking the absolute value squared, interference between individual terms may arise.



An estimation of the order of magnitude of typical
two-photon absorption cross sections can
be obtained by 
assuming only one   intermediate state $\ket{x}$, which is also close to the
final state $\ket{e}$. The two-photon absorption cross section is then related to the product of two one-photon absorption cross sections\footcite{xu97}:
\begin{equation}
\sigma_{TPE} \approx \frac{\sigma_{OPE} \; \sigma_{OPE}}{ \omega}
= \frac{(10^{-17} \text{ cm}^2)^2}{10^{15} \text{ sec}^{-1}} =
10^{-49}\text{ cm}^4\text{sec}^{1} = 10\text{ GM} \quad .
\end{equation}
In a first approximation, the TPE absorption cross section is therefore
proportional to the square of the OPE absorption cross section. The
two-photon absorption cross section $\sigma_{TPE}$ is often given in
of the unit Göppert-Meier (1~GM =
$10^{-50}\text{ cm}^4\text{sec}^{1}$) and typical values are
between 1 and 500~GM. 
Some  values are given  in
table \ref{tab:tpe_sigma}.



\begin{margintable} \renewcommand{\arraystretch}{1.1}
\begin{tabular}{lll}
 Dye & $\lambda_{\text{ex}}$ (nm) & $\sigma_{TPE}$ (GM) \\
\hline Rhodamin B & 840    & $210 \pm 55$ \\
DiI      & 700    & $95 \pm 28$   \\
Fluorescein & 782  &  $38 \pm 9.7$ \\
Coumarin 307 & 776  &  $19 \pm 5.5$   \\
Indo-1   & 700    &  $12 \pm 4$ \\
Bis-MSB & 691    & $6.3 \pm 1.8$ \\
Lucifer Yellow & 860  &  $0.95 \pm 0.3$   \\ 
\end{tabular}
\caption{Two photon absorption cross sections $\sigma_{TPE}$
for dyes at different excitation wavelengths
$\lambda_{\text{ex}}$ (from \cite{xu96mar}). } \label{tab:tpe_sigma}
\end{margintable}


Things get even simpler if 
\emph{no} intermediate state $x$ is assumed, e.g. because all states in
question are energetically too far away. Then the
sum runs only over the states $e$ and $g$, so that
\begin{equation}
\sigma_{TPE} \propto \left| \frac{\mu_{eg}}{\omega} \right|^2 \;
\left| \mu_{ee} -  \mu_{gg} \right|^2 \quad .
\end{equation}
The absorption cross section is therefore proportional to the change in
static dipole moment $\mu_{ii}$ from ground to excited
state. The change of the dipole moment then becomes particularly large,
if charges in the molecule are shifted during excitation.
Efforts to find molecules with particularly large
two-photon absorption cross section,
concentrate on this effect and find values of  $\sigma_{TPE}$ of more than 1000~GM\footcite{albota98}.

 


With two-photon excitation 
states in atoms and molecules can be reached that are not accessible for
 one-photon absorption (OPA).
In the case of TPE  the parity of the
initial state is equal to the final state, whereas it
must be different for OPE.  However, even for
large atoms the selection rules are relaxed, so that the distinction between
different classes of molecular states is not always
 possible. 
 




\section{Stimulated Raman Scattering}


\begin{marginfigure}

\inputtikz{\currfiledir sms_cars}
\caption{ Level scheme }
\end{marginfigure}


As an example, we discuss stimulated Raman scattering to populate a vibrational excited state of the electronic ground state. This follows the same formalism as two-photon absorption, although the second photon is a stimulated emission. The two interactions with a light field lead in both cases to second-order perturbation theory. As intermediate states, we take the electronic excited state and its first vibrational excitation into account. We find interference between these two paths, i.e., the amplitudes have to be added to a common $X_{fi}$ before we take the absolute square of $X_{if}$. In contrast to two-photon absorption, the two interactions happen with   different fields, defined by the frequencies $\omega_1,\omega_2$ and amplitudes $E_1, E_2$. The table summarizes the variables used.

\begin{table}

\begin{tabular}{l|lll|ll}
state & atom & elec. & vibr. & light $\omega_1$ & light
$\omega_2$ \\ \hline%
initial $\ket{i}$ & ground  $\ket{g}$ & 0  & 0 & a  & b  \\%
intermediate $\ket{x}$ & excited  $\ket{e}$& 1 & 0  & a-1 & b \\%
intermediate' $\ket{x'}$ & excited'  $\ket{w}$& 1 & 1& a-1  & b \\ %
final $\ket{f}$ & vibronic  $\ket{v}$ & 0 &1& a-1 &b+1
\end{tabular}
\caption{States of the quantum mechanical system}
\end{table}

The vibrational energy is denoted as $ \Omega_v$. The detuning of the first light field from the electronic resonance is very large compared to the vibrational frequency, i.e.,  $\delta_1 = \omega_{eg} - \omega_1  \gg
\Omega_v$. We write the Frank-Condon-Integral
from vibronic level $a$ in the electronic ground state to $b$ in the
electronic excited state as$F_a^b$. We assume  $F_0^0 = F_1^1 \approx 1$ and $F_0^1 = - F_1^0$. The only transition dipole moment that we need is that for the transition g $\rightarrow$ e, i.e.
 $\mu_{eg}$.

With this, eq. \ref{eq:tpe_pe_xfi} from above becomes 
\begin{equation}
P_v(t)    =   \frac{2 t}{ \Gamma_v}
 \left|
 \frac{ |\mu_{eg}|^2 \, E_1 \, E_2  \, F_0^1}{\hbar^2}%
 \frac{\Omega_v}{(\delta_1)^2 } \right|^2
 \label{eq:tpe_vib_ex_final}
\end{equation}
which agrees\sidenote{The paper makes the distinction between $\omega_{eg}$ and $\omega_{wg}$ which is dropped here.} with the outcome of section 2.3(i) in \cite{Winterhalder11}. 



\begin{table}
\begin{tabular}{ll}
parameter & value \\ \hline
wavelength $\lambda$ & 570 nm \\
fluorescence lifetime $\tau_{fl}$ & 4 ns \\
laser power $P_1 = P_2$ & 100 mW \\
laser repetition rate $f$ & 76 MHz \\
laser pulse width $\tau$ & 1 ps \\
laser spot radius $r$ & 3 $\mu$m \\
Franck-Condon factor $F_0^1$ & 0.5 \\
vibrational frequency $\Omega_v$ & 1500 cm$^{-1}$ \\
vibrational width $\Gamma_v$ & 10 cm$^{-1}$ \\
detuning $\delta_1$ & 8000 cm$^{-1}$
\end{tabular}
\caption{Numerical values which are more optimistic that those in \cite{Winterhalder11}.}
\end{table}


\begin{questions}
\item Derive eq. \ref{eq:tpe_vib_ex_final} from eq. \ref{eq:tpe_pe_xfi}.

\item Calculate the transition dipole moment $\mu_{eg}$ from the fluorescence lifetime $\tau_{fl}$, assuming a quantum efficiency of one.

\item Calculate the field amplitude $ E_1$ assuming a rectangular pulse shape and focus profile.

\item Calculate the population  of the vibrational state at the end of a laser pulse.

\end{questions}


\printbibliography[segment=\therefsegment,heading=subbibliography]
