\renewcommand{\lastmod}{July 1, 2021}
\renewcommand{\chapterauthors}{Markus Lippitz}


\chapter{Four-Wave Mixing}


\section{Tasks}

\begin{itemize}
\item A laser shines on a gold film which generates almost degenerate four-wave mixing signal, i.e., close in frequency but not identical. Predict the FWM spectrum and compare with the experiment (Data by Christoph Schnupfhagn, Bayreuth). You can assume a spectrally constant nonlinear  susceptibility.

\item In a similar experiment, the relative phase between two parts of the laser spectrum is modified by a pulse shaper. Predict the resulting FWM spectrum and compare with the experiment.
\end{itemize}

\section{Experiment}

The laser creates pules that are shorter than 10 fs, or about 250 nm wide. Such pulses are very delicate, as almost all reflecting and transmitting optical elements change the phase differently for the red and the blue part of the laser pulse spectrum. For longer, i.e. spectrally narrower pulses, this does not pose a relevant problem, but for wide pulses, this has to be taken into account. A 'pulse shaper' is a device to compensate these effects and will be discussed in a separate chapter.

Such a pulse shaper allows to generate laser pulses of almost arbitrary amplitude and phase spectrum. Here, we use it to demonstrate four-wave mixing. The data gives the laser intensity spectrum. The phase is either made to be constant or varies linear  in time.
We separate excitation and detection spectrally by a dichroic beam splitter and detect all light below a certain corner wavelength. 

\section{Generation of short pulses}


All non-linear effects  require high
intensities of at least one wave. This 
is relatively
easy to achieve, if one does not require a permanently high laser intensity, but periodic
intensity peaks are sufficient. During these pulses the
intensity  is sufficiently high for nonlinear optics. Between the pulses 
the intensity is so low that no other processes take place. The averaged power of such a laser can therefore
be in the range of about one Watt and still 
peak values of about 100~kW are reached when a pulse duration of
100~fs is compared to a pulse spacing of 13~ns. In
this section we will discuss how such short
laser pulses can be generated.


The starting point is the  wide gain spectrum of a
titanium-sapphire laser. The coupling of the atomic titanium levels to
vibrational states of the sapphire crystal cause a strong
homogeneous broadening of the transitions\footcite[chapter 4]{Rulliere2005}, so that
this type of laser can operate in the wavelength range 700--1000~nm. This allows many longitudinal modes $E_n$  of
electric field in the cavity  to reach the laser threshold. The resonator length $L$ defines the frequency spacing of the
modes. Mode $n$ has the frequency 
\begin{equation}
 \omega_n = n \, \pi c / L \quad.
\end{equation}
The  field of a multi-mode laser is the sum of its
modes:
\begin{equation}
  E(t) = \sum_n E_n(t) = \sum_n \, \hat{E}_n \, e^{i \phi_n} e^{i (c/L) \pi n \, t} \quad .
\end{equation}
The temporal evolution of the initial field is thus  the Fourier transform of the  spectral amplitudes $\hat{E}_n$ multiplied by a phase factor $e^{i \phi_n} $. The pulse period  $T = 2L /c$  corresponds to the round-trip  time in the resonator.
If no further precautions are taken, the
phase $\phi_n$ of each mode take a random  and fluctuating value. This
results\footcite{DielsRudolph1996} in a  pattern of incoherent light, repeating in time with the period $T$.
However, if the individual modes are coupled, i.e., if they 
have a fixed phase relationship to each other,
\begin{equation}
   \delta \phi = \phi_{n+1} - \phi_n = \text{const.} \quad,
   \label{eq:fwm:gl_fwm_phiconst}
\end{equation}
then the temporal evolution of the electric field corresponds to the
Fourier transforms the amplitude distribution. The constant
$\delta \phi$ leads to a shift in  time. A typical resonator length is $L = 2$~m. At
a center wavelength of $\lambda_0 = 800$~nm and a
spectral width of the gain region (limited through 
wavelength-selective  (Lyot) filters) of $\Delta \lambda =
4$~nm, about
\begin{equation}
 N \approx \frac{\delta \lambda}{ \lambda^2} \, \frac{L}{\pi} =
 3980
\end{equation}
modes are above the gain threshold. It is thus justified to assume  a
continuous amplitude distribution $\hat{E}(\omega)$.

As an example, we discuss a Gaussian distribution, others are found in the literature. \footcite{DielsRudolph1996,Rulliere2005}.
We write the spectral amplitude distribution $\hat{E}(\omega)$ as
\begin{equation}
  \hat{E}(\omega) = \hat{E}_0 \, \sqrt{\pi} \, \tau_G \, e^{-
  \frac{1}{4} \, (\omega - \omega_0)^2 \, \tau_G^2} \quad .
\end{equation}
The temporal evolution of the electric field follows from this by
Fourier transformation
\begin{equation}
  \hat {E}(t) = \hat{E}_0 \, e^{- ( t / \tau_G ) ^2}
  \qquad\text{and}\qquad E(t) = \hat{E}(t)\, e^{i \omega_0 \, t} \quad .
\end{equation}
The pulse width $\tau_p$ is the full width at half maximum (FWHM) of the electric field envelop
\begin{equation}
  \tau_p = \sqrt{2 \, \ln 2} \, \tau_G \quad.
\end{equation}
The spectral width of the laser is expressed as FWHM $\Delta
\nu$ of the laser spectrum $S(\omega) =
|\hat{E}(\omega)|^2$ and
\begin{equation}
  \delta \nu = \frac{\delta \omega}{2 \pi} = \frac{\sqrt{8 \, \ln
  2}}{2 \pi \, \tau_G} \quad.
\end{equation}
The Fourier transform relates these two quantities and leads to the so-called \emph{time-bandwidth product}
\begin{equation}
  \delta \nu \, \tau_p = \frac{2 \, \ln 2}{\pi} \approx 0.441 \quad .
\end{equation}
Laser pulses whose time-bandwidth product reach this value
are called \emph{Fourier-limited}. If the available
 spectral bandwidth $\Delta \nu$ is not used
to get the shortest possible pulses, i.e. the smallest $\tau_p$,
then the measured time-bandwidth product exceeds the
value of $0.441$. Provided that pulse shape is Gaussian, the time-bandwidth product tells whether the 
laser system is adjusted optimally.

\begin{questions}
\item All the prefactors in the time-bandwidth product of a Gaussian pulse stem from the use of the FWHM as measure of the spectral and temporal pulse width. Use a $1/e$ width and derive a 'cleaner' form.

\item Assume a few (2 to  20) spectrally equi-distant modes, their amplitudes and phases, and calculate numerically the output field of such a laser. When do you get nice pulses in time? \cite{DielsRudolph1996}, have a picture of this.

\end{questions}



\section{Dispersion}

How does it come that the time-bandwidth product is increased? 
Let us investigate the  influence of a time-dependent phase $\phi(t)$
of the electric field. For simplification we 
assume that $\phi(t)$ is a polynomial in $t$. This means that
\begin{equation}
  E(t) = \hat{E}(t) \, e^{i ( \omega_0 \, t + \phi(t))}
  = \hat{E}(t) \, e^{i ( \omega_0 \, + d\phi(t)/dt) \, t} \, e^{i
  \, \phi_0}  \quad . \label{eq:fwm_field_with_phase}
\end{equation}
As long as $\phi(t)$ is only linearly dependent on the time $t$, the effect is
 only a shift the central frequency $\omega_0$ to
$\omega_0' = \omega_0 + d\phi(t)/dt$.  In frequency space -- through
Fourier transform of the above equation --  this means that the
phase $\phi$ depends only linearly on the frequency $\omega$. This
is what we required already above by  
$\delta \phi =$~const. . For a square
dependence of the phase on the time a temporal variation occurs
in the central frequency $\omega_0$ of the pulse. One  speaks
of \emph{chirp}, because in the acoustic analogy of a
rising or falling tone sequence. Square
(and higher) dependence of the phase on time (and thus on
the frequency) is determined by the group velocity dispersion
(GVD) in optical elements,
e.g. glasses, but also mirrors. This is associated with the
first and second derivative of the refractive index with 
wavelength. Assuming $\phi = - a \,
t^2 / \tau_G^2$ we then get
\begin{equation}
  \hat{E}(t) = \hat{E}_0 e^{- (1 + i \, a) \, ( t / \tau_G ) ^2} \label{eq:fwm_chirp}
\end{equation}
and for the time-bandwidth product
\begin{equation}
  \delta \nu \, \tau_p = \frac{2 \, \ln 2}{\pi} \, \sqrt{1 + a^2} \approx
  0.441 \, \sqrt{1 + a^2} \quad ,
\end{equation}
as the spectral bandwidth $\Delta \nu$ is just increased by a factor of
$\sqrt{1 + a^2}$. Higher order polynomial terms in the time-dependence of the phase $\phi$ are discussed in the chapter on  pulse shaping.



Group velocity dispersion (GVD) cannot be avoided  in an
optical resonator. How transmitting and
also reflective elements change the phase of the
electromagnetic wave  depends strongly on the wavelength. In order to still achieve an optimal, i.e. Fourier limited pulse
the entire GVD must be as wavelength independent as possible. The influence of the individual elements should therefore 
compensate  each other. In glasses, the positive second derivative of the refractive index
is dominating  the GVD. To compensate for this, a section with
negative GVD is needed in the resonator. This can be  achieved by two prisms as shown in figure
\ref{fig:fwm_prism}. Two equal
prisms are placed in front of a mirror such 
that all spectral components hit the mirror perpendicularly
and  the resonator is closed.  

 Detailed calculations \footcite{DielsRudolph1996}
show that the pure geometric path length difference causes 
negative GVD. The path inside the prisms contributes  positive GVD.  By varying the position $z$ one can obtain almost arbitrary values of net GVD of this prism system.


\begin{marginfigure}
   \inputtikz{\currfiledir  prism_v2}
\caption{This prism sequence allows to adjust the group velocity dispersion by the prism position along $z$. }
\label{fig:fwm_prism}
\end{marginfigure}


\begin{questions}
\item Draw / calculate the laser pulse  given by eq. \ref{eq:fwm_chirp} and its properties so that one can understand why this is called a 'chirp'.
\end{questions}



\section{Autocorrelation}

A pulse width $\tau_P$ in the range of a few pico- or femto-seconds cannot be measured electronically, as the involved frequencies  would be too high. The idea is to measure the pulse length optically, using the  laser pulse to sample itself. The beam is divided into two parts, one of which is delayed with respect to the other before they are overlaid again.
 Times in the 
range of femtoseconds correspond to lengths in the range of
micrometers, which can be easily controlled. A detector that responds quadratically in intensity (not field!) is able to measure  overlap of two halves $a$ and $b$ as
\begin{equation}
 \ \text{after each other} \qquad a^2 + b^2 \neq (a+b)^2 \qquad
 \text{simultaneously.}
\end{equation}
Often  frequency doubling is used for this purpose by focusing the
recombined beam onto a frequency doubling crystal.
The detector is sensitive only for light at half the wavelength. With variable delay $\tau$ the time-integrated signal is sufficient, and  the temporal resolution
of the detector is  irrelevant. One thus measures
\begin{equation}
  G(\tau) = \left< I(t) \times I(t-\tau) \right> \quad.
\end{equation}
The function $G(\tau)$ is called as intensity autocorrelation function. It is always symmetrical ($G(\tau) = G(-\tau)$) and
is relatively insensitive to the actual pulse shape.  One has to assume a pulse as Gaussian
 or
$\text{sec}^2$-shaped to calculate\footcite{DielsRudolph1996} the pulse length $\tau_{P} $ from the 
FWHM of the autocorrelation
$\tau_{AC}$ 
\begin{align}
  \tau_{P} &= \tau_{AC} / \sqrt{2} &\quad& \text{for Gauss pulses,} \\
           &= \tau_{AC} / 1,543 && \text{for $\text{sec}^2$-shaped pulses.}
\end{align}


\begin{questions}
\item Why is the intensity autocorrelation symmetric in delay $\tau$ ?
\end{questions}



		
\section{Four-wave mixing}

Let us now turn to the spectroscopic consequences of the phase dispersion over the spectral width of a laser pulse. We use the example of four-wave mixing. In general, a third-order nonlinearity can be written as
\begin{equation}
  P^{(3)}_i = \epsilon_0 \, 
    \sum_{j,k,l} \chi^{(3)}_{ijkl} E_j(\omega_1) E_k(\omega_2) E_l(\omega_3)  
\end{equation}		
where three optical waves produce a nonlinear polarization, which radiates a new optical field\sidenote{Similar to the chapter on the free induction decay}. This field is sometimes detected directly, sometimes interfered with  a fourth field acting as local oscillator. A \emph{third-order} nonlinearity describes thus the mixing of \emph{four} waves.
		
		
Several different processes are 	summarized as four-wave mixing (FWM) effects and depicted in figure \ref{fig:fwm_processes}.  The generated nonlinear polarization oscillates at the sum of the three input frequencies. Complex conjugate input fields are again counted as negative frequencies and depicted as downward arrows in Fig. \ref{fig:fwm_processes}.
	
\begin{figure}
\inputtikz{\currfiledir  chi3-processes_v2}

\caption{Sketches of  $\chi^{(3)}$ processes:
%
third-harmonic generation (THG), pump-probe spectroscopy (PP), coherent anti-Stokes Raman scattering (CARS), and  non-degenerate four-wave mixing (FWM).
%
 All processes start and end at a real state  that can be populated (solid horizontal line). Intermediate states can be virtual states (dashed horizontal line). The interaction between light and matter occurs in time from left to right within each diagram. The last downward arrow symbolizes the emission process. Other downward  arrows symbolize incoming complex-conjugate fields that count negative when calculating the emission frequency.
\label{fig:fwm_processes}}
\end{figure}

In third-harmonic generation, one optical wave at frequency $\omega$ generates a new optical field at frequency $3 \omega$: 
%
\begin{equation}
P^{(THG)}(3 \omega) = \epsilon_0 \chi^{(3)} E(\omega)^3  \quad .
\label{eq:fwm_chi3-thg}
\end{equation}
%
This is a coherent process, i.e., the phase of the new field is related to the phase of the fundamental wave, but homodyne interference is in contrast to linear absorption not possible. The detected signal intensity $I^{(THG)}$ is thus proportional to the third power of the fundamental intensity $I(\omega)$
\begin{equation}
I^{(THG)} \propto \left| P^{(THG)}(3 \omega) \right|^2 = \left| \epsilon_0 \chi^{(3)} E(\omega)^3 \right|^2 \propto I(\omega)^3 \quad .
\label{eq:fwm_intensity-thg}
\end{equation}


In pump-probe spectroscopy, a pump-beam is absorbed and its influence on the absorption of a second, probe beam is monitored. Here it is the intensity of the pump beam, not the field amplitude that enters. The probe beam appears only linear in field amplitude as in linear absorption spectroscopy.
All together pump-probe spectroscopy can be written as
%
\begin{align}
P^{(pp)}(\omega_{\text{probe}}) = &
\epsilon_0 \chi^{(3)} \, \ E(\omega_{\text{pump}}) \ E^*(\omega_{\text{pump}})  \, E(\omega_{\text{probe}})  \nonumber \\
= &
\epsilon_0 \chi^{(3)} \, \left| E(\omega_{\text{pump}})  \right|^2 \, E(\omega_{\text{probe}})  
\quad .
\label{eq:fwm_chi3-pp}
\end{align}
%
From this notation, it is obvious that pump-probe spectroscopy does not depend on the phase relation between pump- and probe-beam.
The generated nonlinear polarization oscillates at the frequency of the incoming probe beam which again allows homodyne detection. The observed signal is thus proportional to the amplitude of the interference between nonlinear polarization and probe field and not to the square of the nonlinear polarization itself as in third-harmonic generation. The  signal  scales linearly both  in pump and in probe power.


In coherent anti-Stokes Raman scattering (CARS) the pump field also enters twice, as in pump-probe spectroscopy, but here it enters twice with the same phase. Both pump arrows point up in Figure \ref{fig:fwm_processes}. The nonlinear polarization is given by 
%
\begin{equation}
P^{(CARS)}(\omega_{\text{CARS}}) = 
\epsilon_0 \chi^{(3)} \, \ E(\omega_{\text{pump}}) \ E(\omega_{\text{pump}})  \, E^*(\omega_{\text{Stokes}}) 
\label{eq:fwm_chi3-cars}
\end{equation}
%
and oscillates at a new frequency $\omega_{CARS} = 2 \omega_{pump} - \omega_{Stokes}$. Either one detects its square (similar to third-harmonic generation) or one supplies an additional field acting as local oscillator to produce a signal that is linear in the generated polarization. The intermediate state of lowest energy in Fig. \ref{fig:fwm_processes} is a vibrational state of a molecule. It enhances the efficiency of this process which can also take place without a real state at this energy. 

In the most general case of four-wave mixing all three input waves differ in frequency. The examples above are cases of degenerate four-wave mixing (DFWM) where some frequencies coincide. Here, I use the term 'four-wave mixing' only for the non-degenerate case and stick to the more specialized terms otherwise. An experimentally accessible variant of four-wave mixing is depicted in Fig. \ref{fig:fwm_processes} where three similar but not identical waves produce a fourth wave which again is spectrally close to the input waves. This process scales linearly in all three input powers and also the phase relation between all waves is relevant.

Note that for a given material the value of $\chi^{(3)} $ depends on all three incoming frequencies $\omega_1, \omega_2, \omega_3$ and the outgoing frequency $\omega_4$, which is written as  $\chi^{(3)}(\omega_4; \omega_1, \omega_2, \omega_3)$. 

\begin{questions}
\item Convince yourself that the direction of the arrows in  Fig. \ref{fig:fwm_processes} fits to the equations given in this chapter.

\item Draw a level scheme for a hypothetical $\chi^{(5)}$ process and give the equation to calculate its intensity.
\end{questions}


\section{Spectral interference}

How does now the spectral phase enter? With the broad spectrum of  a laser pulse, many different combinations of $\omega_1, \omega_2, \omega_3$ lead to the same outgoing frequency $\omega_4$. All these paths interfere with each other so that with each frequency $\omega_i$ also its phase $\phi_i$ enters.

Neglecting the tensorial nature of $\chi$ we can write
\begin{equation}
  P^{(3)}(\omega_4) = \epsilon_0 \, 
    \int_{\omega_1, \omega_2} \chi^{(3)}(\omega_4; \omega_1, \omega_2, \omega_3) \,  E(\omega_1) E(\omega_2) E(\omega_3) \, d\omega_1 \, d\omega_2  
\end{equation}	
where $\omega_3$ is calculated at each point of the integral such that energy conservation is fulfilled, e.g. $\omega_3 = \omega_4 - \omega_1 - \omega_2$. Again, the above rule for negative frequencies, complex conjugates, and downward arrows applies.

The spectral dependence of $\chi^{(3)}(\omega_4; \omega_1, \omega_2, \omega_3)$ is determined by the process that causes the nonlinearity. As in linear spectroscopy, the spectral shape of  $\chi^{(3)}$ is related to the response of the system in time domain, its pulse response\sidenote{Multiplication in frequency domain becomes by Fourier transform a convolution in time domain.} A fast system has thus a spectrally flat susceptibility. In the limit of instantaneous response, the nonlinear susceptibility is spectrally constant.
		
When assuming a flat nonlinear susceptibility is too much of an approximation, one can make use of Miller's rule\footcite{Boyd2008,Miller64,Obermeier18}, which connects linear and nonlinear optical properties
\begin{equation}
  \chi^{(3)}(\omega_4; \omega_1, \omega_2, \omega_3) \propto  \chi^{(1)}(\omega_1) \,   \chi^{(1)}(\omega_2)  \,   \chi^{(1)}(\omega_3)  \,   \chi^{(1)}(\omega_4)  \quad .
 \end{equation}
 		
\begin{questions}
\item A laser pulse has a spectral width of $10$ nm. How broad is the spectrum of the second- and third-harmonic that is created by this laser, assuming nonlinear susceptibilities that are spectrally flat in the relevant spectral range?
\end{questions}
 
\printbibliography[segment=\therefsegment,heading=subbibliography]

