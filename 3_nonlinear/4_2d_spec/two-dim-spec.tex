\renewcommand{\lastmod}{June 21, 2024}
\renewcommand{\chapterauthors}{Markus Lippitz}


\chapter{Two-Dimensional Spectroscopy}


\section{Tasks}

\begin{itemize}
\item Find all possible Feynman diagrams for a 3-level V-system, by hand or with a computer. Group them by their phase matching conditions. Assume some parameters and plot 2D spectra. The e-learning course has some useful Matlab routines.

\item Investigate the possibility of 2D spectroscopy to find coupling between systems with the example of a 'diamond' level system. A ground state $\ket{0}$ is connected to a top-most excited state $\ket{3}$ via two 'paths' passing state $\ket{1}$ or $\ket{2}$. The energetic position of state $\ket{3}$ is $E_3 = E_1 + E_2 + \delta$ ($E_0= 0$). Discuss why $\delta = 0$ corresponds to two uncoupled systems. Plot the rephased and non-rephased spectra and their sum for zero and non-zero values of $\delta$.


\end{itemize}

\section{Interaction Picture}

A more complete introduction to 2D spectroscopy and double-sided Feynman diagrams can be found in \cite{Hamm-dummies}, which is a short version of \cite{HammZanni2011}, which is a short version of \cite{Mukamel1995}.


In the Schrödinger picture, all the time-dependence sits in the wave function
\begin{eqnarray}
 \hat{H} & = & const. \\
 \frac{d}{dt} \ket{\psi(t)} &=& - \frac{i}{\hbar} \, \hat{H} \, \ket{\psi(t)} \\
 \braket{\hat{A}} &=& \braket{\psi(t) \, | \,\hat{A} \, |\, \psi(t) }  \quad .
\end{eqnarray}
The solution to the differential equation for $\ket{\psi(t)}$ is
\begin{equation}
 \ket{\psi(t)} =  \exp \left(- \frac{i}{\hbar} \hat{H} \, t \right) \,\ket{\psi(0)} \quad .
\end{equation}


In the Heisenberg picture, all the time-dependence sits in the operator
\begin{eqnarray}
 \ket{\psi_H} &=& const. \\
 \frac{d}{dt} \hat{A}_H(t) &=& - \frac{i}{\hbar} \, \left[ \hat{H} , \hat{A}_H(t) \right] \\
 \braket{\hat{A}_H} &=& \braket{\psi \, | \,\hat{A}_H (t) \, |\, \psi }  \quad .
\end{eqnarray}
The solution to the differential equation for  $\hat{A}_H(t)$ is
\begin{equation}
 \hat{A}_H(t) = \exp \left(+ \frac{i}{\hbar} \hat{H} \, t \right) \, \hat{A}(0) \, \exp \left(- \frac{i}{\hbar} \hat{H} \, t \right) \quad .
\end{equation}
At time $t=0$ both pictures come together,i.e., $\hat{A}_H(0) = \hat{A}$ and $\psi_H = \psi(0)$ and of course both pictures give the same result  for the expectation value  of an operator.

In  this chapter, we use the  interaction picture, which is in a sense half-way between Schrödinger and Heisenberg: We assume a small perturbation $\hat{H}'(t)$ in addition to an unperturbed $\hat{H}_0$
\begin{equation}
 \hat{H}(t) = \hat{H}_0 + \hat{H}'(t) \quad . \label{eq:2d_hamilton_perturbation}
\end{equation}
The effect of  $\hat{H}_0$ is to modify the operators, while $\hat{H}'(t)$ modifies the wave functions
\begin{eqnarray}
  \frac{d}{dt} \ket{\psi(t)} &=& - \frac{i}{\hbar} \, \hat{H}' \, \ket{\psi(t)}  \label{eq:2d_schroedinger_interaction}\\
 \frac{d}{dt} \hat{A}(t) &=& - \frac{i}{\hbar} \, \left[ \hat{H}_0 , \hat{A}(t) \right] \\
 \braket{\hat{A}} &=& \braket{\psi(t) \, | \,\hat{A} (t) \, |\, \psi(t) }  \quad .
\end{eqnarray}


In the interaction picture we can now integrate up the Schrödinger equation \ref{eq:2d_schroedinger_interaction} and transform back to the Schrödinger picture. To keep things simple, I will only show the results when transformed back. Intermediate steps are given in \cite{Hamm-dummies}.


We assume  that the system experiences a perturbation by the operator $\hat{H}'$ at time $\tau$, but evolves unperturbed in the remaining parts of the time interval from $t_0$ to $t$.  We have made the approximation that only one interaction takes place in this interval, which is only the first element of a power series. With that we get
\begin{equation}
\ket{\psi(t)}  
  \approx  \ket{\psi(t_0)} - \frac{i}{\hbar} \int_{t_0}^t d \tau \,  U_0( t , \tau) \,\hat{H}'(\tau) \,  U_0 (\tau , t_0) \, \ket{\psi(t_0)}  \label{eq:2d_time_evol}
\end{equation}
with the operator for free (= unperturbed) time evolution $U_0$ from time $t_1$ to $t_2$
\begin{equation}
 U_0(t_2, t_1) = \exp \left(- \frac{i}{\hbar} \hat{H}_0 \, ( t_2- t_1) \right) \quad .
\end{equation}



We are particularly interested in the time evolution of the density matrix according to the Liouville-von Neumann equation. Starting from $\tau = t_0$ we take more and more interactions into account. $\rho^{(n)}$ describes the evolution of the density matrix assuming $n$ interactions between $\tau = t_0$ and $\tau = t$
\begin{equation}
\rho(t) = \rho_0  + \sum_n  \rho^{(n)} (t) \quad ,
\end{equation}
where $\rho_0$ is the equilibrium density matrix which does not evolve in time.
The term for $n=1$ reads, equivalent to eq.  \ref{eq:2d_time_evol}  above,
\begin{equation}
 \rho^{(1)} (t) = - \frac{i}{\hbar} \int_{t_0}^t \, d\tau \, U_0(t, t_0) \, \left[ \hat{H}'_I(\tau) \, , \, \rho_0 \right] \, U_0^\star (t, t_0) \quad ,
\end{equation}
where $\hat{H}'_I(\tau) $ is the perturbation operator in the interaction picture
\begin{equation}
\hat{H}'_I(\tau) =  U_0^\star (\tau , t_0) \, \hat{H}'(\tau) \, U_0 (\tau , t_0) \quad .
\end{equation}
Multiple interactions now result in cascaded commutators, i.e.
\begin{equation}
 \rho^{(2)} (t) = \left(- \frac{i}{\hbar}\right)^2   \int_{t_0}^t \, d\tau_2   \int_{t_0}^{\tau_2} \, d\tau_1 \, U_0(t, t_0) \, \left[ \hat{H}'_I(\tau_2) \, , \, \left[ \hat{H}'_I(\tau_1) \, , \, \rho_0 \right] \right] \, U_0^\star (t, t_0) \quad .
\end{equation}

In spectroscopy, we are interested in a perturbation operator $\hat{H}'$ that contains an $\mu \cdot E$ term for the incoming, exciting field. We see that $\rho^{(n)}$ depends on $n$ optical fields, so we identify these processes with the $n$-th order nonlinear processes described by the nonlinear susceptibility $\chi^{(n)}$. This nonlinear susceptibility generates a nonlinear polarization $P^{(n)}$. In quantum mechanics, this is the expectation value of the dipole operator 
\begin{equation}
P^{(n)} (t) = \braket{ \mu \, \rho^{(n)}(t) } \quad .
\end{equation}
For third-order nonlinear effects, we thus get
\begin{equation}
 \left(- \frac{i}{\hbar}\right)^3   \iiint E(\tau_3) E(\tau_2) E(\tau_1)
 \braket{ \mu \, 
  \left[ \mu_3 , \, \left[ \mu_2 , \, \left[ \mu_1 , \, \rho_0 \, \right] \right]  \right] } \quad .
  \label{eq:2d-p3-commutator}
\end{equation}
The cascaded commutator expands to $2^3=8$ terms, where one half is the complex conjugate of the other, effectively giving $4$ terms.

In the experiment, the electric fields for a third-order nonlinearity would be generated by three different laser pulses. In general, each electric field $E(\tau_i)$ at time $\tau_i$ could be a superposition of the three pulses. We make the approximation that the pulses do not overlap in time (time ordering). We also assume that the pulses are shorter than the time scale of the system\footnote{and thus shorter than the delay between the pulses}, but much longer than the oscillation period of the light wave. Each electric field can thus be described by two terms of the form $\exp( \pm i (\omega t - k r) )$. This gives a total of $2^3=8$ terms for the field, multiplied by the 4 terms from the commutator gives 32 terms.
In the remainder, we try to keep track of all these terms.



\section{Double-Sided Feynman Diagrams}

\begin{marginfigure}
\inputtikz{\currfiledir feynman_example}
\caption{
Example of a Double-Sided Feynman Diagram.}
\label{fig:2d_example_feynman}
\end{marginfigure}

The idea is to describe the evolution of the density matrix when interacting with multiple optical fields.
Time runs from bottom to top. The two vertical lines represent the ket (left) and bra (right) parts of the density matrix element. The quantum numbers are written between the lines and we start from the ground state $\ket{0}\, \bra{0}$. Diagonal arrows represent the light fields interacting with the system. Arrows pointing in increase the quantum number, arrows pointing out decrease the quantum number. The last, top arrow is drawn with a dashed or wavy line and represents the dipole operator $\mu$ outside the commutator in eq. \ref{eq:2d-p3-commutator}. This is the radiating nonlinear polarization. By convention, it is on the left side, pointing up left. This convention fixes the mirror symmetry that all schemes would also exist in a complex conjugate form. At the end, after the nonlinear polarization has been emitted, all coherences must have disappeared and the system must be in a population state, i.e. both quantum numbers must be identical.\sidenote{but not necessarily zero}.



We still have the choice of sign in $E \propto \exp( \pm i (\omega t - k r) )$. We write this as $\exp(- i (\omega t - k r) )$ and choose the sign of $\omega$ and $k$.
A positive sign is represented by an arrow pointing to the right, a negative sign to the left. The positive frequency is therefore absorption at the left vertical line, or emission at the right vertical line, and vice versa. This is where phase matching, i.e. $\Delta k = 0$, and the rotating wave approximation meet. This allows us to calculate the spatial direction of the radiating polarization by summing all wave vectors $\mathbf{k}$ of the interactions (non-wavy arrows).
Thus, the number of arrows pointing left and right must be the same when the special radiation polarization arrow is included.

Each diagram has a general sign, derived from the minus sign in the commutators in eq. \ref{eq:2d-p3-commutator}. The sign is $(-1)^m$, where $m$ is the number of interactions on the right side of the diagram.

The diagram represents a path between elements of the density matrix. As we saw in the Rabi Oscillations chapter when the density matrix was introduced, a light field first causes a non-zero value of the corresponding off-diagonal element. A field acting from the left produces $\ket{1} \bra{0}$, while a field acting from the right produces $\ket{0} \bra{1}$. Figure \ref{fig:2d_liouville_path} shows possible paths in Liouville space.

\begin{marginfigure}
\inputtikz{\currfiledir liouville_path}

\caption{
Two different paths  through the density matrix by applying four times the dipole operator. The red path corresponds to Fig. \ref{fig:2d_example_feynman}.}
\label{fig:2d_liouville_path}
\end{marginfigure}

At the end, we are interested in $P^{(3)}(t)$ or $P^{(3)}(\omega)$. We can easily obtain it by multiplying the elements along the path. In the time domain\footcite{Mukamel1995,HammZanni2011,Hamm-dummies}, each interaction with an optical field contributes
\begin{equation}
 - \frac{i}{\hbar} \, \mu_{ba} \, E_i(t)  \, e^{-i \omega_i \, t} \quad ,
\end{equation}
where $a$ ($b$) is the quantum number before (after) the interaction. If the arrow points to the left, the field is taken complex-conjugate (and thus with negative frequency, see below). The time interval $t_i$ between two interactions contributes 
%
\begin{equation}
 e^{-i \, \omega_{ab} \, t_i - \Gamma_{ab} \, t_i} \quad ,
\end{equation}
%
where we have assumed the state to be $\ket{a} \bra{b}$, and  $\hbar \omega_{ab} = E_a - E_b$ so that $\omega_{ab}  = - \omega_{ba} $. The last outgoing arrow of the radiating polarization only contributes a dipole moment, as this is the dipole moment multiplied to the left of the cascaded commutators.

In frequency domain\footcite{Boyd2008,Shen2003}, we get for the interactions
\begin{equation}
 - \frac{i}{\hbar} \, \mu_{ba} \, E_i(\omega_i)   \quad .
\end{equation}
The phase evolution during the time interval $t_j$ needs a Fourier transform.  The Fourier-conjugate variable\footcite{Tokmakoff09} to $t_j$  is $\Omega_j$, which is the sum of all laser frequencies up to interaction $j$, i.e.,
\begin{equation}
 \Omega_j = \sum_{k=1}^j \pm \omega_k \quad .
\end{equation}
The sign is again taken from the direction of the arrow (positive to the right). We thus get by a Fourier transform\sidenote{Note that $P=0$ for $t_j < 0$}  along $t_j$ 
\begin{equation}
 e^{ (-i \omega_{ab} - \Gamma_{ab}) t_j }  \quad \rightarrow \quad
 \frac{i }{\Omega_j - \omega_{ab}  + i \Gamma_{ab} }  \quad .
\end{equation}




\section{Example: Pump-Probe Spectroscopy}

Let us consider pump-probe spectroscopy as an example. As we saw in the chapter on four-wave mixing, pump-probe spectroscopy is a third-order nonlinear optical process of the form
\begin{equation}
 P^{(3)} =  \chi^{(3)} E_\text{pump} \, E_\text{pump}^\star \, E_\text{probe} \quad .
\end{equation}
For simplicity, we assume that the probe pulse comes after the pump pulse. The nonlinear emission must be in the direction of the probe pulse to act as a local oscillator. Since the emitting arrow points to the left, the probe arrow must point to the right. The sub-schemes A and B fulfill this requirement.

\begin{figure}
\inputtikz{\currfiledir feynman_pp_elements}

\caption{
Building blocks for the interaction of the probe pulse (panels A and B) and the pump-pulse (panels 1 to 4) in a pump-probe experiment.}
\label{fig:2d_feynman_pp_elements}
\end{figure}

The direction of the pump beam must cancel, because $E_\text{pump} $ enters only as absolute-squared. This means that the two pump arrows must point in different directions. The sub-schemes 1 to 4 fulfill this requirement.

\begin{figure}
\inputtikz{\currfiledir feynman_pump_probe}
\caption{
Six double-sided Feynman diagrams describe pump-probe spectroscopy.}
\label{fig:2d_feynman_pump_probe}
\end{figure}

Two of the eight combinations are not allowed. The variants A3 and A4 would describe the emission from the ground state $\ket{0} \bra{0}$, which of course is forbidden. Variants A2 and A2 describe stimulated emission, B1 and B2 excited state absorption, and B3 and B4 ground state bleaching.

From the diagrams we can now read the nonlinear polarization by multiplying the contribution of each interaction and the times $t_i$ between them. For diagrams A1 and A2 we get
\begin{eqnarray}
P^{(3)}_{A1} & =& \frac{i}{\hbar^3} \,  E_1 E_2^\star E_3 \, \mu_{10}^4 \,
e^{ (-i \omega_{10} - \Gamma_{10}) t_1 } \,
e^{  - \Gamma_{11} t_2 } \,
e^{ (-i \omega_{10} - \Gamma_{10}) t_3 }  \\
P^{(3)}_{A2} & =& \frac{i}{\hbar^3} \,  E_1^\star E_2 E_3 \, \mu_{10}^4 \,
e^{ (+i \omega_{10} - \Gamma_{10}) t_1 } \,
e^{  - \Gamma_{11} t_2 } \,
e^{ (-i \omega_{10} - \Gamma_{10}) t_3 }  
\end{eqnarray}
where only the direction of the phase rotation changes during the time interval $t_1$ between the first and second interaction (and the field, which is complex conjugate). In pump-probe spectroscopy we assume that both $E_1$ and $E_2$ originate from the same laser pulse, the pump pulse, and that we can neglect $t_1$ because the pulse is short. So for stimulated emission (SE) we get
\begin{eqnarray}
P^{(3)}_{SE} & =& 2 \frac{i}{\hbar^3} \,  | E_\text{pump}|^2 E_\text{probe} \, \mu_{10}^4 \,
e^{  - \Gamma_{11} t_2 } \,
e^{ (-i \omega_{10} - \Gamma_{10}) t_3 } \quad .
\end{eqnarray}
In a similar way we get for excited state absorption (ESA) and ground state bleaching (GSB)
\begin{eqnarray}
P^{(3)}_{ESA} & =& -2 \frac{i}{\hbar^3} \,  | E_\text{pump}|^2 E_\text{probe} \, \mu_{10}^2 \mu_{21}^2 \,
e^{  - \Gamma_{11} t_2 } \,
e^{ (-i \omega_{21} - \Gamma_{21}) t_3 }  \\
P^{(3)}_{GSB}  & =& 2 \frac{i}{\hbar^3} \,  | E_\text{pump}|^2 E_\text{probe} \, \mu_{10}^4 \,
e^{ (-i \omega_{10} - \Gamma_{10}) t_3 }  \quad .
\end{eqnarray}
The minus sign in the ESA term is due to the global sign of the diagram, since ESA diagrams have only one interaction from the right. The GSB term does not include the decay of the population during the time interval $t_2$, since the ground state does not decay. If we neglect the population decay during $t_2$, the terms for GSB and SE are identical. The nonlinear polarization is the sum of these three terms.

As discussed in the chapter on free induction decay, a polarization (or coherence) is a source of a radiated field that interferes with a local oscillator, if present. In pump-probe spectroscopy, the probe field acts as a local oscillator. The pump-induced variation of the probe power is caused by the cross term of the local oscillator and the signal field.
\begin{equation}
 \Delta P = \frac{1}{2} \epsilon_0 \int_\text{pulse}  2 E_\text{probe} \; \Im (E_\text{signal} ) \, dt 
 = N L \frac{k}{2}  \int_\text{pulse}   E_\text{probe} \; \Im ( P^{(3)} ) \, dt \quad .
\end{equation}

For pump-probe spectroscopy we get for the differential spectrum, assuming a spectrally flat probe pulse with $\omega_3 = \omega_\text{probe}$
\begin{equation}
\frac{\Delta P (\omega_3)} {P_\text{probe}(\omega_3)} 
\propto
%=  N L \frac{k}{2 \sqrt{2 \pi} \hbar^3}  | E_\text{pump}|^2  \left( 
- \frac{4 \, \mu_{10}^4 \, \Gamma_{10} } { (\omega_3 - \omega_{10})^2 + \Gamma_{10}^2} 
+ \frac{2 \, \mu_{10}^2  \, \mu_{21}^2 \, \Gamma_{21} } { (\omega_3 - \omega_{21})^2 + \Gamma_{21}^2}  \quad .
%\right)
\end{equation}


\section{Example: Two-Photon Absorption}

The formalism of double-sided Feynman diagrams allows us to obtain results from perturbation theory by just reading the diagram. For comparison, we turn back to the example of \cite{Winterhalder11}, discussed in the last chapter. We read the $\nu g $-coherence as
\begin{equation}
\rho_{\nu g}(t_1, t_2) = \frac{(-i)^2}{\hbar^2} \, \mu_{eg} \, \mu_{\nu e} \, E_1 \, E_2 \,
e^{-i \omega_{eg} t_1 - \Gamma_{eg} t_1} \, 
e^{-i \omega_{\nu g} t_2 - \Gamma_{\nu g} t_2}  \quad .
\end{equation}
Fourier transformation leads to
\begin{eqnarray}
\rho_{\nu g}(\omega_1, \omega_2) &=& \frac{(-i)^2}{\hbar^2} \, \mu_{eg} \, \mu_{\nu e} \, E_1 \, E_2 \,
    \frac{i }{\Omega_1 - \omega_{eg}  + i \Gamma_{eg} } 
  \, \frac{i }{\Omega_2  -\omega_{\nu g}  + i \Gamma_{\nu g} } \\
  %
  &=& \frac{\mu_{eg} \, \mu_{\nu e} \, E_1 \, E_2 }{\hbar^2} \,
  \frac{1 }{\omega_1 - \omega_{eg}  + i \Gamma_{eg} } 
  \, \frac{1}{\omega_1 - \omega_2  -\omega_{\nu g}  + i \Gamma_{\nu g} } 
\end{eqnarray}
which is eq. 1b in \cite{Winterhalder11}.
 

\section{2D spectroscopy}


In contrast to pump-probe spectroscopy, we now assume that all three interactions occur at three different times, so that we do not neglect any time interval $t_i$. However, all three pulses come from the same laser, have the same spectrum, but different propagation directions $k_i$. In 2D spectroscopy, the nonlinear polarization is usually detected by interfering with a fourth laser pulse that acts as a local oscillator\footcite{Ogilvie15,Scholes13}. In some variants, the fourth pulse is used to generate a population that is then detected by electron or fluorescence emission. Also in some cases not the propagation direction $k_i$ but a individual phase $\phi_i$ is used to distinguish the pulses. In the following discussion we stick to the most common or 'traditional' approach.

\begin{marginfigure}
\inputtikz{\currfiledir fwm_phase_matching}

\caption{The phase-matching direction selects the observed process. Interference with a local oscillator (LO) after reflection at a beam splitter (BS) allows detection (det) of a complex-valued polarization.}
\label{fig:2d_setup_phasematch}
\end{marginfigure}


How can we do spectroscopy if all laser pulses have the same spectrum? The approach is the same as in Fourier transform (infrared) spectroscopy. Two laser pulses separated by a short time interval interfere with each other, resulting in a sinusoidal modulation of the combined spectrum. The Fourier transform of the time interval into frequency space then provides the frequency axis for spectroscopy. In 2D spectroscopy, we Fourier transform along the first and third time intervals. During these times, the system is in a coherent superposition between two states described by the density matrix element $\ket{a}\bra{b}$ with $a \neq b$. The second, non-Fourier transformed time interval is called the population time, since the system is (typically) in a population state $\ket{a}\bra{a}$. Note that the third time interval must have an end. It ends with the emission of the nonlinear polarization. This is not controlled by a laser pulse on the system. However, if we use the fourth laser pulse as a local oscillator, we can decide when to amplify and then detect the emitted polarization. Or we can let physics do the Fourier transform and detect the emitted radiation without further interference by a spectrometer that does not need time resolution.\sidenote{We detect $|P|^2$ this way, not the complex $P$.}

We obtain 2D spectra, along an excitation and a detection frequency, for a fixed population time interval $t_2$. One can imagine obtaining similar spectra by scanning both pump and probe wavelengths in a pump-probe experiment. One big difference is that the pulses in such a pump-probe experiment would have to be quite long, because they have to be spectrally narrow. This limits the time resolution in $t_2$. With the four-pulse experiment, we get both high time resolution and spectral information.

What features do we find in a 2D spectrum? Along the diagonal it contains the linear absorption spectrum. We pump and probe the same molecules. If our ensemble of molecules is inhomogeneously broadened, i.e. the molecules differ in absorption wavelength much more than the width of their individual absorption spectrum, then the peak in the 2D spectrum will be elongated along the diagonal and narrowed along the antidiagonal. We pump a sub-ensemble of molecules, and only these will not be able to absorb the probe pulse. 2D spectroscopy can thus distinguish between homogeneous and inhomogeneous broadening, similar to spectral hole burning.


\begin{marginfigure}
\inputtikz{\currfiledir inhom_broadening}

\caption{An inhomogeneous ensemble of molecules gives rise to an elongated peak in the 2D spectrum. Along the antidiagonal direction it has the \emph{homogeneous} linewidth.}\label{fig:2d_inhom_broadening}
\end{marginfigure}



2D spectroscopy can also find coupled (or interacting) systems. With linear spectroscopy, we only find the energy difference between the initial and final states. If we have two peaks in an absorption spectrum, we cannot tell if there are three or four states involved. We could have two different molecules (4 states) or a V level system in one molecule (3 states). 2D spectroscopy can resolve this question: In the coupled case, we find cross peaks at the other two corners of the square formed by the two transition frequencies. At these points we pump one transition and find an influence on the other transition. This is only possible if these transitions are somehow coupled. Molecule A would not care if molecule B was excited.

\begin{figure}
\inputtikz{\currfiledir sketch_cross_peaks}

\caption{Cross-peaks outside the diagonal (blue circles) signal coupling between systems. This information is not present in the linear absorption spectrum.}
\label{fig_2d_crosspeak}
\end{figure}


\section{Rephasing}

The set of all allowed double-sided Feynman diagrams for a given quantum mechanical system can be grouped by their phase matching condition, i.e. by the signs in front of the $\pm k_i$. An important property of these groups of diagrams is whether rephasing occurs or not. The question is how the coherences during interval one and three are related. If the system is in state $\ket{a}\bra{b}$ during interval one and then in state $\ket{b}\bra{a}$ during interval three, this is called rephasing. Both intervals cancel their phase contribution to $P^{(3)}$, since
\begin{equation}
  e^{- i \omega_{ba} \, t_1} \,   e^{- i \omega_{ab} \, t_3} =   e^{- i  \omega_{ba} \, ( t_1 - t_3)} \quad .
\end{equation}
This is important because the transition energy $\hbar \omega_{ba}$ varies between different molecules in an ensemble and fluctuates over time for a single molecule. In rephasing diagrams, the influence of this inhomogeneity is canceled out. This is very similar to echo formation in pulsed NMR, where the sign of the phase factor represents the direction of rotation. Rephasing diagrams let the phase rotate first in one direction, then in the opposite direction, so that all the 'runners' come together again.\footnote{See the 'runner' example for NMR echoes if this sounds strange to you.} When you add the rephased and non-rephased 2D spectra together, you get a pure absorption spectrum with narrow peaks and no side effects from the Fourier transform.



\printbibliography[segment=\therefsegment,heading=subbibliography]
